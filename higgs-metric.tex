\documentclass[12pt,letterpaper,reqno]{article}

% \usepackage{mathtools}
\usepackage{epsfig}
\usepackage{amsmath}
\usepackage{amssymb}
\usepackage{amsthm}
\usepackage{indentfirst}
\usepackage{xspace}
\usepackage{multirow}
\usepackage{hyperref}
\usepackage{xcolor}
\usepackage{verbatim}
\usepackage[letterpaper,margin=1in,headheight=15pt]{geometry}
\usepackage{mathpazo}
\usepackage{tikz-cd}
\usepackage{booktabs}
\usepackage{framed}
\usepackage{float}
\usepackage{thmtools}
\usepackage{dashrule}
\usepackage[missing=]{gitinfo2}
\usepackage{fancyhdr}

\definecolor{darkblue}{rgb}{0.1,0.1,0.7}
\definecolor{darkred}{rgb}{0.5,0.1,0.1}
\definecolor{darkgreen}{rgb}{0.0,0.42,0.06}
\hypersetup{colorlinks=true,urlcolor=darkred,linkcolor=darkblue,citecolor=darkred}
\definecolor{shadecolor}{rgb}{0.85,0.85,0.85}

% Bibliography formatting
\usepackage[bibstyle=numeric,defernumbers=true,maxnames=20,uniquename=init,backend=biber,sorting=none,bibencoding=utf8,casechanger=latex2e,backref=true]{biblatex}

\DeclareNameAlias{sortname}{first-last}

\DeclareFieldFormat{url}{\url{#1}}
\DeclareFieldFormat[article]{pages}{#1}
\DeclareFieldFormat[inproceedings]{pages}{\lowercase{pp.}#1}
\DeclareFieldFormat[incollection]{pages}{\lowercase{pp.}#1}
\DeclareFieldFormat[article]{volume}{\textbf{#1}}
\DeclareFieldFormat[article]{number}{(#1)}
\DeclareFieldFormat[article]{title}{\MakeCapital{#1}}
\DeclareFieldFormat[inproceedings]{title}{#1}
\DeclareFieldFormat{shorthandwidth}{#1}

% Don't use "In:" in bibliography. Omit urls from journal articles.
\DeclareBibliographyDriver{article}{%
  \usebibmacro{bibindex}%
  \usebibmacro{begentry}%
  \usebibmacro{author/editor}%
  \setunit{\labelnamepunct}\newblock
  \MakeSentenceCase{\usebibmacro{title}}%
  \newunit
  \printlist{language}%
  \newunit\newblock
  \usebibmacro{byauthor}%
  \newunit\newblock
  \usebibmacro{byeditor+others}%
  \newunit\newblock
  \printfield{version}%
  \newunit\newblock
%  \usebibmacro{in:}%
  \usebibmacro{journal+issuetitle}%
  \newunit\newblock
  \printfield{note}%
  \setunit{\bibpagespunct}%
  \printfield{pages}
  \newunit\newblock
  \usebibmacro{eprint}
  \newunit\newblock
  \printfield{addendum}%
  \newunit\newblock
  \usebibmacro{pageref}%
  \usebibmacro{finentry}}

% Remove dot between volume and number in journal articles.
\renewbibmacro*{journal+issuetitle}{%
  \usebibmacro{journal}%
  \setunit*{\addspace}%
  \iffieldundef{series}
    {}
    {\newunit
     \printfield{series}%
     \setunit{\addspace}}%
  \printfield{volume}%
%  \setunit*{\adddot}%
  \printfield{number}%
  \setunit{\addcomma\space}%
  \printfield{eid}%
  \setunit{\addspace}%
  \usebibmacro{issue+date}%
  \newunit\newblock
  \usebibmacro{issue}%
  \newunit}


% Bibliography categories
\def\makebibcategory#1#2{\DeclareBibliographyCategory{#1}\defbibheading{#1}{\section*{#2}}}
\makebibcategory{books}{Books}
\makebibcategory{papers}{Refereed research papers}
\makebibcategory{chapters}{Book chapters}
\makebibcategory{conferences}{Papers in conference proceedings}
\makebibcategory{techreports}{Unpublished working papers}
\makebibcategory{bookreviews}{Book reviews}
\makebibcategory{editorials}{Editorials}
\makebibcategory{phd}{PhD thesis}
\makebibcategory{subpapers}{Submitted papers}
\makebibcategory{curpapers}{Current projects}

\setlength{\bibitemsep}{2.65pt}
\setlength{\bibhang}{.8cm}
\renewcommand{\bibfont}{\small}

\renewcommand*{\bibitem}{\addtocounter{papers}{1}\item \mbox{}\hskip-0.85cm\hbox to 0.85cm{\hfill\arabic{papers}.~~}}
\defbibenvironment{bibliography}
{\list{}
  {\setlength{\leftmargin}{\bibhang}%
   \setlength{\itemsep}{\bibitemsep}%
   \setlength{\parsep}{\bibparsep}}}
{\endlist}
{\bibitem}

\newenvironment{publications}{\section{\LARGE Publications}\label{papersstart}\vspace*{0.2cm}\small
\titlespacing{\section}{0pt}{1.5ex}{1ex}\itemsep=0.00cm
}{\label{papersend}\addtocounter{sumpapers}{-1}\refstepcounter{sumpapers}\label{sumpapers}}

\def\printbib#1{\printbibliography[category=#1,heading=#1]\lastref{sumpapers}}

% Counters for keeping track of papers
\newcounter{papers}\setcounter{papers}{0}
\newcounter{sumpapers}\setcounter{sumpapers}{0}
\def\lastref#1{\addtocounter{#1}{\value{papers}}\setcounter{papers}{0}}

% theorem environments
\declaretheoremstyle[spaceabove=0.25cm,spacebelow=0.25cm,notefont=\normalfont\bfseries, notebraces={(}{)}]{theorem}
\declaretheoremstyle[spaceabove=0.25cm,spacebelow=0.25cm,bodyfont=\normalfont,notefont=\normalfont\bfseries, notebraces={(}{)}]{noital}
\declaretheoremstyle[spaceabove=0.25cm,spacebelow=0.25cm,bodyfont=\normalfont\color{darkgreen},notefont=\normalfont\bfseries, notebraces={(}{)}]{green}
\declaretheoremstyle[spaceabove=0.25cm,spacebelow=0.25cm,bodyfont=\normalfont,notefont=\normalfont\bfseries,qed=$\qedsymbol$,notebraces={(}{)}]{proofstyle}

\declaretheorem[name=Theorem,numberwithin=section,style=theorem]{thm}
\declaretheorem[name=Proposition,sibling=thm,style=theorem]{prop}
\declaretheorem[name=Conjecture,sibling=thm,style=theorem]{conj}
\declaretheorem[name=Corollary,sibling=thm,style=theorem]{cor}
\declaretheorem[name=Lemma,sibling=thm,style=theorem]{lem}
\declaretheorem[name=Remark,sibling=thm,style=theorem]{remark}
\declaretheorem[name=Definition,sibling=thm,style=noital]{defn}
\declaretheorem[name=Example,sibling=thm,style=noital]{example}
\declaretheorem[name=Exercise,numberwithin=section,style=green]{exercise}
\declaretheorem[name=Proof,style=proofstyle,numbered=no]{pf}

\numberwithin{equation}{section}


% macros for convenience
\newcommand{\tops}{\texorpdfstring}

\newcommand{\nid}{\noindent}

\newcommand{\fa}{{\mathfrak a}}
\newcommand{\fp}{{\mathfrak p}}
\newcommand{\fk}{{\mathfrak k}}
\newcommand{\fg}{{\mathfrak g}}
\newcommand{\ft}{{\mathfrak t}}
\newcommand{\fh}{{\mathfrak h}}
\newcommand{\fn}{{\mathfrak n}}
\newcommand{\fq}{{\mathfrak q}}
\newcommand{\fm}{{\mathfrak m}}
\newcommand{\fr}{{\mathfrak r}}
\newcommand{\fsu}{{\mathfrak{su}}}
\newcommand{\fu}{{\mathfrak u}}
\newcommand{\fG}{{\mathfrak G}}

\newcommand{\cC}{\ensuremath{\mathcal C}}
\newcommand{\cG}{\ensuremath{\mathcal G}}
\newcommand{\cB}{\ensuremath{\mathcal B}}
\newcommand{\cL}{\ensuremath{\mathcal L}}
\newcommand{\cS}{\ensuremath{\mathcal S}}
\newcommand{\cF}{\ensuremath{\mathcal F}}
\newcommand{\cK}{\ensuremath{\mathcal K}}
\newcommand{\cZ}{\ensuremath{\mathcal Z}}
\newcommand{\cM}{\ensuremath{\mathcal M}}
\newcommand{\cN}{\ensuremath{\mathcal N}}
\newcommand{\cO}{\ensuremath{\mathcal O}}
\newcommand{\cH}{\ensuremath{\mathcal H}}
\newcommand{\cX}{\ensuremath{\mathcal X}}
\newcommand{\cY}{\ensuremath{\mathcal Y}}
\newcommand{\cA}{\ensuremath{\mathcal A}}
\newcommand{\cI}{\ensuremath{\mathcal I}}
\newcommand{\cW}{\ensuremath{\mathcal W}}

\newcommand{\bbR}{\ensuremath{\mathbb R}}
\newcommand{\bbQ}{\ensuremath{\mathbb Q}}
\newcommand{\bbZ}{\ensuremath{\mathbb Z}}
\newcommand{\bbC}{\ensuremath{\mathbb C}}
\newcommand{\bbN}{\ensuremath{\mathbb N}}
\newcommand{\bbH}{\ensuremath{\mathbb H}}
\newcommand{\bbP}{\ensuremath{\mathbb P}}
\newcommand{\bbI}{\ensuremath{\mathbb I}}

\newcommand{\half}{\ensuremath{\frac{1}{2}}}
\newcommand{\qtr}{\ensuremath{\frac{1}{4}}}
\newcommand{\bq}{{\mathbf q}}
\newcommand{\N}{{\mathcal N}}
\newcommand{\F}{{\mathcal F}}
\newcommand{\HH}{{\mathcal H}}
\newcommand{\LL}{{\mathcal L}}
\newcommand{\RR}{{\mathcal R}}
\newcommand{\V}{{\mathcal V}}
\newcommand{\dirac}{\!\!\not\!\partial}
\newcommand{\Dirac}{\!\!\not\!\!D}
\newcommand{\cE}{{\mathcal E}}
\newcommand{\vs}{\not\!v}
\newcommand{\kahler}{K\"ahler\xspace}
\newcommand{\kq}{/\!\!/}
\newcommand{\kql}[1]{/\!\!/\!\!_#1\,}
\newcommand{\hk}{hyperk\"ahler\xspace}
\newcommand{\Hk}{Hyperk\"ahler\xspace}
\newcommand{\hkq}{/\!\!/\!\!/\!\!/}
\newcommand{\hkql}[1]{/\!\!/\!\!/\!\!/\!\!_#1\,}
\newcommand{\del}{\ensuremath{\partial}}
\newcommand{\delbar}{\ensuremath{\overline{\partial}}}
\newcommand{\I}{{\mathrm i}}
\newcommand{\J}{{\mathrm j}}
\newcommand{\K}{{\mathrm k}}
\newcommand{\e}{{\mathrm e}}
\newcommand\bid{{\mathbf 1}}
\newcommand{\de}{\mathrm{d}}
\newcommand{\ab}{\mathrm{ab}}
\newcommand{\sing}{\mathrm{sing}}
\newcommand{\reg}{\mathrm{reg}}
\newcommand{\vol}{\mathrm{vol}}
\renewcommand{\sf}{\mathrm{sf}}
\newcommand{\inst}{\mathrm{inst}}
\newcommand{\eff}{\mathrm{eff}}
\newcommand{\dR}{\mathrm{dR}}
\newcommand{\closed}{\mathrm{closed}}
\newcommand{\exact}{\mathrm{exact}}
\newcommand{\gauge}{\mathrm{gauge}}
\newcommand{\flavor}{\mathrm{flavor}}

\newcommand{\abs}[1]{\lvert#1\rvert}
\newcommand{\norm}[1]{\lVert#1\rVert}
\newcommand{\IP}[1]{\langle#1\rangle}
\newcommand{\DIP}[1]{\langle\!\langle#1\rangle\!\rangle}
\newcommand{\dwrt}[1]{\frac{\partial}{\partial#1}}
\newcommand{\eps}{\epsilon}
\newcommand{\simarrow}{\xrightarrow\sim}

\newcommand{\bSigma}{{\overline \Sigma}}
\newcommand{\vphi}{{\vec\phi}}

\newcommand{\mmaref}[1]{}

\newcommand{\ti}[1]{\textit{#1}}
\newcommand{\tb}[1]{\textbf{#1}}

\DeclareMathOperator{\ad}{ad}
\DeclareMathOperator{\im}{Im}
\DeclareMathOperator{\re}{Re}
\DeclareMathOperator{\Tr}{Tr}
\DeclareMathOperator{\End}{End}
\DeclareMathOperator{\ParEnd}{ParEnd}
\DeclareMathOperator{\SParEnd}{SParEnd}
\DeclareMathOperator{\Hom}{Hom}
\DeclareMathOperator{\Aut}{Aut}
\DeclareMathOperator{\Sym}{Sym}
\DeclareMathOperator{\Lie}{Lie}
\DeclareMathOperator{\diag}{diag}
\DeclareMathOperator{\Bun}{Bun}
\DeclareMathOperator{\Vect}{Vect}
\DeclareMathOperator{\Span}{Span}
\DeclareMathOperator{\grad}{grad}
\DeclareMathOperator{\rank}{rank}
\DeclareMathOperator{\ind}{ind}
\DeclareMathOperator{\coker}{coker}
\DeclareMathOperator{\Jac}{Jac}
\DeclareMathOperator{\Pic}{Pic}
\DeclareMathOperator{\Prym}{Prym}
\DeclareMathOperator{\Hol}{Hol}
\DeclareMathOperator{\gr}{gr}
\DeclareMathOperator{\Res}{Res}
\DeclareMathOperator{\Tot}{Tot}
\DeclareMathOperator{\pdeg}{pdeg}
\DeclareMathOperator{\Gr}{Gr}
\DeclareMathOperator{\DT}{DT}

\newcommand{\SO}{\mathrm{SO}}
\newcommand{\SL}{\mathrm{SL}}
\newcommand{\GL}{\mathrm{GL}}
\newcommand{\PSL}{\mathrm{PSL}}
\newcommand{\SU}{\mathrm{SU}}
\newcommand{\Sp}{\mathrm{Sp}}
\newcommand{\U}{\mathrm{U}}

\newcommand{\insfig}[2]{

\medskip
\noindent
\begin{minipage}{\linewidth}

\makebox[\linewidth]{\includegraphics[keepaspectratio=true,scale=#2]{figures/#1-crop.pdf}}

\end{minipage}
\medskip

}


% \newcommand{\insfig}[2]{\begin{figure}[htbp] \centering \includegraphics[scale=#2]{figures/#1-crop.pdf} \label{fig:#1} \end{figure}}
% syntax: \insfig{name}{0.5}{caption}

\newcommand{\fixme}[1]{{\color{orange}{[#1]}}}
\newcommand{\currentposition}{{\color{blue} \noindent\makebox[\linewidth]{\hdashrule{\paperwidth}{1pt}{3mm}}}}

% \mathtoolsset{showonlyrefs}

\bibliography{higgs-metric}

\begin{document}

\pagestyle{fancy}
\lhead{{\tiny \color{gray} \tt \gitAuthorIsoDate}}
\chead{\tiny \ti{Metric on moduli of Higgs bundles}}
\rhead{{\tiny \color{gray} \tt \gitAbbrevHash}}
\renewcommand{\headrulewidth}{0.5pt}


\begin{center}
\tb{Metric on moduli of Higgs bundles} \\
Andrew Neitzke
\end{center}

{These are notes for a lecture series on moduli of Higgs bundles,
originally delivered at the mini-workshop ``Gauge theory and Geometry in G\"ottingen''
March 1-4, 2018, subsequently modified and updated for the 2024 Simons Workshop in Mathematics.
% Please send corrections/improvements to
% \begin{center}
% \small \tt\href{mailto:neitzke@math.utexas.edu}{neitzke@math.utexas.edu}
% \end{center}
% or as pull requests to the source repository hosted at
% \begin{center}
% \small \url{http://github.com/neitzke/higgs-metric}
% \end{center}
}

% \tableofcontents
% \renewcommand{\listtheoremname}{Quick reference}
% \listoftheorems[onlynamed]

% \newpage


\section{Introduction}

The aim of these lectures is to describe a \ti{conjectural} 
approach to more explicitly understanding the \hk metrics on 
moduli spaces of Higgs bundles. 
I will specialize to $\SL(2)$-Higgs bundles, but the
picture is similar for any reductive $G$.

Basic philosophy: replace $\SL(2)$-Higgs bundles and flat connections over a Riemann
surface $C$ by $\GL(1)$-Higgs bundles and flat connections over a branched double cover $\Sigma_{\phi_2}$
of $C$, called \ti{spectral curve}. The surprise is how well this 
works. Indeed, the conjecture says that the metric on $\cM$
is constructed from two ingredients:
\begin{enumerate}
  \item The \ti{periods} $Z_\gamma$ of the spectral curves $\Sigma_{\phi_2}$,
  % \insfig{higgs-metric-4}{0.9}
  \item A collection of integer ``Donaldson-Thomas invariants''
  $\DT(\gamma)$,
  which count (tropicalizations of) special Lagrangian discs
  in $T^*C$ with boundary on $\Sigma_{\phi_2}$.
  % \insfig{higgs-metric-5}{0.8}
\end{enumerate}

Although ultimately the metrics we want to understand
are smooth and have no obvious integrality properties,
this way of understanding them involves Donaldson-Thomas
invariants, cluster transformations, 
wall-crossing phenomena.
It is most digestible near asymptotic infinity, but in principle
gives information everywhere on $\cM$.

% So, what we will describe is a conjectural recipe which uses $Z_\gamma$
% and $\DT(\gamma)$ to build the \hk metric on $\cM$.

% If one sets all $\DT(\gamma) = 0$ in this recipe,
% one obtains a simple and explicit \hk metric $g^\sf$ on $\cM_\reg$, 
% the ``semiflat metric,'' so called
% because it is flat (and translation invariant) on the torus
% fibers. $g^\sf$ however does not extend over $\cM_\sing$, so it cannot
% be equal to Hitchin's metric on $\cM$.
% The effect of the nonzero $\DT(\gamma)$
% is to add corrections which break the translation invariance.
% Away from $\cM_\sing$, these corrections are of the order 
% $\e^{-2M}$ where $M$ is the area of the smallest 
% special Lagrangian disc, i.e. we predict
% \begin{equation} \label{eq:conjecture-intro}
%   g = g^\sf + O(\e^{-2M}).
% \end{equation}
% In particular, if we follow some path to infinity in $\cM$
% along which $M \to \infty$, we expect to see $g$
% converging to $g^\sf$.
% On the other hand, as we approach $\cM_\sing$ the effect
% of these corrections is large: it smoothes out the singularity
% of $g^\sf$.\footnote{In some cases the \hk metric $g$ on $\cM$ actually has
% singularities. The singularities of $g$ are milder than those of $g^\sf$ --- in particular, the singularities of $g$ occur 
% in complex codimension at least $2$. In these cases,
% the corrections do not completely \ti{smooth} the singularity of $g^\sf$, but they
% still \ti{improve} it.}

The strategy of the lectures will be roughly:
\begin{enumerate}
  \item The moduli space $\cM$, its basic features and \hk structure; the weak form of the metric conjecture (now almost proven).
  \item The strong form of the conjecture (idea: calculate the nonabelian Hodge map), 
  and the available evidence that the conjecture is correct.
\end{enumerate}

This is a review of joint work with Davide Gaiotto and Greg Moore,
follow-up work with Lotte Hollands and David Dumas; closely related 
work by Mazzeo-Swoboda-Weiss-Witt, Fredrickson, Mochizuki; 
inspired by Fock-Goncharov, Kontsevich-Soibelman, Hitchin,
Corlette, Donaldson, Simpson, Biquard-Boalch.


\subsection{References}

The conjecture reviewed in these notes is mostly contained in 
the papers \cite{Gaiotto:2008cd,Gaiotto:2009hg},
which are joint work of mine with Davide Gaiotto and Greg
Moore.
In \cite{notes-hk} I reviewed some parts of the conjecture,
focusing on the abstract construction of \hk metrics from a
special \kahler base and Donaldson-Thomas invariants;
in these lectures I focus more on the examples provided by
moduli spaces of Higgs bundles,
and even more specifically, on one specific moduli space of
Higgs bundles with irregular singularity.

These works depend on many prior developments
in physics and mathematics. Here I can only single out a few
which were of singular importance (for more, see the references
in \cite{Gaiotto:2008cd,Gaiotto:2009hg}):
\begin{itemize}
\item The work \cite{Gaiotto:2008cd} 
originated in an attempt to understand the physical
meaning of the remarkable wall-crossing formula for generalized
Donaldson-Thomas invariants, given by Kontsevich-Soibelman \cite{ks1}.
\item Many of the key constructions in 
\cite{Gaiotto:2008cd} can
be understood as infinite-dimensional analogues 
of constructions used by Cecotti-Vafa and Dubrovin in 
$tt^*$ geometry \cite{Cecotti:1993rm,MR1213301}, with
additional inspiration from work of Bridgeland and 
Toledano Laredo \cite{stab-stokes}.
\item The application to Hitchin systems
in \cite{Gaiotto:2009hg} depended importantly on
the work of Fock-Goncharov on moduli spaces
of local systems over surfaces \cite{MR2233852},
as well as the foundational work
of Hitchin \cite{MR89a:32021} and Corlette, Donaldson, Simpson
\cite{MR965220,MR887285,MR944577} on Higgs bundles without
singularities, Simpson's extension to 
Higgs bundles with regular 
singularities \cite{hbnc}, and Biquard-Boalch for Higgs
bundles with wild ramification \cite{wnh}.
\end{itemize}

\section{Background on Hitchin system}

\subsection{The unpunctured case}

Let me first recall the simplest case to define (though not the simplest
case to study!) The most fundamental reference is \cite{MR89a:32021}.
A very useful review can be found in
\cite{Wentworth2014} and references therein.

We fix a compact Riemann surface $C$ of genus $g_C \ge 2$.

\begin{defn}[$\SL(2)$-Higgs bundles] \label{def:higgs}
A $\SL(2)$-\ti{Higgs bundle} is a pair $(E, \varphi)$, where:
\begin{itemize}
  \item $E$ is a holomorphic vector bundle of rank $2$ over $C$, with $\det E = \cO$,
  \item $\varphi$ is a holomorphic section of $\End E \otimes K_C(P)$, with $\Tr \varphi = 0$.
\end{itemize}
\end{defn}
With respect to a local trivialization of $E$, then, $\varphi$ is represented
by a traceless $2 \times 2$ matrix whose entries are holomorphic $1$-forms.

\begin{defn}[Stability of $\SL(2)$-Higgs bundles]
We say $(E,\varphi)$ is \ti{stable} if, for all $\varphi$-invariant $E' \subset E$, we have $\deg E' < 0$.
We say $(E,\varphi)$ is \ti{strictly polystable} if it is a direct sum of two degree-zero $\varphi$-invariant line bundles.
We say $(E,\varphi)$ is \ti{polystable} if it is either stable or strictly polystable.
\end{defn}

\begin{prop}[Moduli space of Higgs bundles]
There is a moduli space $\cM = \cM(G,C)$ parameterizing
polystable $\SL(2)$-Higgs bundles $(E,\varphi)$ 
up to equivalence. 
$\cM$ is a manifold away
from the locus of strictly-polystable Higgs bundles.
It has complex dimension $6 g_C - 6$.
It carries a natural complex structure $I_1$ and holomorphic symplectic form $\Omega_1$.
\end{prop}

The holomorphic symplectic form $\Omega_1$ comes from the fact 
that variations of the bundle $E$ are valued in $H^1(\End E)$,
while variations of the Higgs field $\varphi$ are
valued in $H^0(\End E \otimes K_C)$,
and the two are Serre dual.

% \begin{remark}[Dimension of $\cM$]
% The complex dimension of $\cM$ is
% \begin{equation} \label{eq:dim-M}
%   \dim_\bbC \cM = \begin{cases} (2g_C-2) K^2 + 2, & \quad G = \U(K), \\ (2g_C-2) (K^2-1), & \quad G = \SU(K). \end{cases}
% \end{equation}
% From now on we write $\dim_\bbC \cM = 2r$.
% \end{remark}

% \begin{example}[The abelian case]
% $G = \U(1)$ is the abelian case, in which $\dim_\bbC \cM = 2g_C$,
% and $\cM = T^* \Jac C$, the cotangent
% bundle to a compact complex torus.
% This is the only example of $\cM$ which is so ``linear'' in nature:
% for nonabelian $G$ the space $\cM$ will be much more interesting.
% \end{example}

% \begin{example}[Lowest-dimensional nonabelian case] 
% The simplest nonabelian case is
% $G = \SU(2)$ and $g_C = 2$, in which case
% $\dim_\bbC \cM = 6$. 
% \end{example}



\subsection{The Hitchin map}

Next we exhibit $\cM$ as a complex integrable system,
i.e. a holomorphic Lagrangian fibration over a base 
$\cB$.

Given a Higgs bundle $(E,\varphi) \in \cM$ and $z \in C$,
we can consider the eigenvalues of $\varphi(z)$. As $z$ varies
these sweep out a curve $\Sigma$:
% \begin{equation}
%   \Sigma = \{ (z,\lambda): \det(\lambda - \varphi(z)) = 0\} \subset \Tot[K_C].
% \end{equation}
% We call $\Sigma$
% the \ti{spectral curve} associated to the Higgs 
% bundle $(E,\varphi)$.
% Concretely it is of the form
\begin{equation}
  \Sigma = \{ (z,\lambda): \lambda^2 + \phi_2(z) = 0\} \, \subset T^*C \, ,
\end{equation}
where
\begin{equation}
  \phi_2 = - \frac12 \Tr \varphi^2 \, .
\end{equation}
The projection $\rho: \Sigma \to C$ given by $\rho(z,\lambda) = z$ is a branched double cover,
ramified at the zeroes of $\phi_2$.
% The eigenvalues of $\varphi(z)$ are $\pm \sqrt{- \phi_2(z)}$.

\begin{defn}[Hitchin base and Hitchin map]
Define the \ti{Hitchin base} $\cB = H^0(C, K_C^2)$.
The \ti{Hitchin map} is the map $\pi: \cM \to \cB$ given by
\begin{equation}
  (E, \varphi) \mapsto \phi_2 = - \frac12 \Tr \varphi^2 \, .
\end{equation}
\end{defn}

\begin{prop}[Hitchin map has Lagrangian fibers] The fibers
$\cM_{\phi_2} = \pi^{-1}(\vphi)$ 
are compact complex Lagrangian subsets of $(\cM, I_1, \Omega_1)$. (In particular, $\dim_\bbC \cB = \half \dim_\bbC \cM$.)
\end{prop}

We can say more precisely what the fibers are,
over most of the Hitchin base:

\begin{defn}[Singular locus and smooth locus]
The \ti{singular locus} $\cB_\sing \subset \cB$ is the set of $\vphi \in \cB$
for which $\Sigma_{\vec \phi}$ is singular.
  % $\cB_\sing$ has complex codimension $1$ in $\cB$.
The \ti{smooth locus} is $\cB_\reg = \cB \setminus \cB_\sing$.
It consists of those $\phi_2$ which have only simple zeroes ($4 g_C - 4$ of them).
Also let $\cM_\reg = \pi^{-1}(\cB_\reg)$.
\end{defn}

\begin{prop}[Fibers of the Hitchin map over $\cB_\reg$]
Suppose $\vphi \in \cB_\reg$.
Then $\cM_{\phi_2}$ is a torsor over the compact complex torus $\Prym (\rho: \Sigma_{\phi_2} \to C)$. 
After choosing spin structures on 
$C$ and $\Sigma_{\phi_2}$, we can identify $\cM_{\phi_2}$ with the space of flat $\U(1)$-connections $\nabla$ over $\Sigma_{\phi_2}$, equipped with a trivialization of $\det \rho_* \nabla$.
\end{prop}

So we reach the following picture: a point $\vphi \in \cB_\reg$
gives a smooth spectral curve $\Sigma_{\phi_2}$; the torus
$\cM_{\phi_2}$ is a space of flat $\U(1)$-connections
over $\Sigma_{\phi_2}$.
% \begin{remark}[More precise picture of the fibers
% when $G = \U(K)$]
% A more precise description would say exactly \ti{which} torsor 
% $\cM_{\phi_2}$ is. For example, when $G = \U(K)$,
% $\cM_{\phi_2}$ is the moduli space of holomorphic line bundles
% over $\Sigma_{\phi_2}$ with degree $-K(K-1)(g_C-1)$.
% After choosing spin structures on $C$ and on $\Sigma$
% we can shift the degree to zero.
% So then we can identify $\cM_{\phi_2}$ with the space of
% flat $U(1)$-connections over $\Sigma_{\phi_2}$.
% \end{remark}
When $\vphi \in \cB_\sing$, $\cM_{\phi_2}$ is compact,
but generally singular.

% \begin{remark}[Concrete description of the singular locus]
% What is $\cB_\sing$ concretely?
% The branch locus of the covering $\rho: \Sigma_{\phi_2} \to C$ 
% is the zero locus of the discriminant $\Delta_{\vphi}$ of the equation \eqref{eq:spectral-coeffs}. $\Delta_{\phi_2}$ 
% is a holomorphic section of $K_C(P)^{K(K-1)}$.
% $\vphi \in \cB_\reg$ iff $\Delta_{\phi_2}$ has only simple zeroes; in this case
% it has $K(K-1)(2g_C+\abs{P}-2)$ of them, and the genus of $\Sigma_{\phi_2}$ is
% \begin{equation}
% g_\Sigma = 1 + K^2(g_C - 1) + \half K(K-1) \abs{P}.
% \end{equation}
% \end{remark}


% \begin{example}[A one-dimensional Hitchin base]
% Suppose $G = \SU(2)$, $\abs{P} = 4$, $g_C = 0$.
% \begin{itemize}
%   \item
% Suppose $m^\bbC = 0$. Then $\cB$ is a complex vector space of dimension $1$,
% and $\cB_\sing \subset \cB$ is the set containing only the point
% $\vphi = 0$, so
% the only singular fiber is $\cM_{\vphi = 0}$.
% For generic $m^\R$,
% $\cM_{\vphi =0}$ consists of five $\bbC \bbP^1$'s
% arranged in an affine $D_4$ configuration.

% \item Suppose $m^\bbC \neq 0$. Then 
% $\cB$ is a complex affine space of dimension $1$.
% If $m^\bbC$ is generic, $\cB_\sing \subset \cB$ consists
% of $6$ points, and the fiber $\cM_{\phi_2}$ over any $\vphi \in \cB_\sing$ is
% a nodal torus. (For special choices of $m^\bbC$ some of these discriminant points 
% may collide.)
% \end{itemize}
% \end{example}

% \insfig{higgs-metric-2}{0.85}

% \fixme{also do the 3-punctured $SU(3)$ case?}

% We should emphasize that this one-dimensional example can lead to the
% wrong mental picture about the generic case: generally,
% when $m^\bbC = 0$, $\cB_\sing$ is some codimension-$1$ cone inside $\cB$, and in particular,
% when $\dim \cB > 1$, $\cB_\sing$ is not compact. For $m^\bbC \neq 0$, $\cB_\sing \subset \cB$
% is not a cone anymore, but near asymptotic infinity of $\cB$,
% it still looks asymptotically like a cone.



\subsection{The \hk metric}

A key fact about $\cM$ is that it carries a canonically
defined \hk metric $g$. However, $g$ is not easily
written in closed form.
To construct $g$, one needs to consider \ti{Hitchin's equation}:
given a Higgs bundle $(E,\varphi)$ this is a PDE
for a Hermitian metric $h$ in $E$, written
\begin{equation} \label{eq:hitchin-equation}
  F_{D_h} + [\varphi, \varphi^{\dagger_h}] = 0.
\end{equation}
Here $D_h$ denotes the Chern connection in $(E,h)$,
the unique $h$-unitary connection compatible with
the holomorphic structure of $E$.

Now there is the following key theorem:
\begin{thm}[Existence of harmonic metrics] \label{thm:hitchin-existence}
The equation \eqref{eq:hitchin-equation} has a solution $h$ for each $(E,\varphi)$;
this $h$ is unique up to scalar multiple.
We call $h$ the \ti{harmonic metric}.
\end{thm}

Using \autoref{thm:hitchin-existence}
one can define Hitchin's metric on $\cM$, as follows.
Given a tangent vector $v$ to $\cM$ whose norm we wish to calculate, we represent $v$ by a family of Higgs bundles
$(E_t,\varphi_t)$, with harmonic metrics $h_t$.
Identifying the underlying Hermitian bundles with a single $(E,h)$
we have an arc of unitary connections $D_t$ and skew-Hermitian 
Higgs fields $\Phi_t = \varphi_t - \varphi_t^\dagger$ 
on $(E,h)$, determined up to gauge transformations i.e. 
automorphisms of $(E,h)$.
In particular, differentiating at $t = 0$ gives a pair
\begin{equation}
\frac{\de}{\de t}\Bigg\rvert_{t=0} (D_t, \Phi_t) = 
(\dot A, \dot \Phi) \in \Omega^1(\fsu(E)) \oplus \Omega^{1}(\fsu(E)),
\end{equation}
defined up to gauge transformations.
Then the norm of $v$ is the $L^2$ norm
\begin{equation} \label{eq:l2-metric-def}
  g(v,v) = \int_C \norm{\dot A}^2 + \norm{\dot \Phi}^2
\end{equation}
where for $(\dot A, \dot \Phi)$ we choose the representative
\ti{minimizing} the norm.

\begin{remark}[\Hk quotient] I have not really explained
why the metric $g$ constructed in this way turns out to 
be \hk, or even \kahler. The most conceptual explanation
of this comes by viewing the construction in terms of 
an infinite-dimensional \hk quotient.
This was explained by
Hitchin in \cite{MR89a:32021}.
% In the case of parabolic Higgs bundles which we
% are considering here,
% the construction of $\cM$ and $g$ by \hk quotient
% was given by Konno in \cite{MR1206652}.

\end{remark}

% \subsection{Hyperkahler structure}

% Although it is not obvious from the way I have described it
% here, the metric $g$ on $\cM$ is in fact \hk.
% In particular, it carries symplectic structures 
% $\omega_1$, $\omega_2$, $\omega_3$:
% \fixme{review the basic structure}
% \fixme{check signs}
% \begin{subequations} \label{eq:symplectic-forms-higgs}
% \begin{align}
%   \omega_1((\dot A_1, \dot \Phi_1),(\dot A_2, \dot \Phi_2)) &= \int_C \Tr(- \dot A_1 \wedge \dot A_2 + \dot \Phi_1 \wedge \dot \Phi_2), \\
%   \omega_2((\dot A_1, \dot \Phi_1),(\dot A_2, \dot \Phi_2)) &= \int_C \Tr(\dot \Phi_1 \wedge \star \dot A_2 - \dot A_1 \wedge \star \dot \Phi_2), \\
%   \omega_3((\dot A_1, \dot \Phi_1),(\dot A_2, \dot \Phi_2)) &= \int_C \Tr(\dot\Phi_1 \wedge \dot A_2 + \dot A_1 \wedge \dot \Phi_2).
% \end{align}
% \end{subequations}
% and complex structures
% $I_1$, $I_2$, $I_3$:
% \begin{subequations}
% \begin{align}
%  I_1(\dot A, \dot \Phi) &= (\star \dot A, - \star \dot \Phi), \\
%  I_2(\dot A, \dot \Phi) &= (- \dot \Phi, \dot A), \\
%  I_3(\dot A, \dot \Phi) &= (- \star \dot\Phi, - \star \dot A).
% \end{align}
% \end{subequations}







% In particular, $\cM$ is noncompact.

\subsection{The semiflat metric}

Now we want to describe a first approximation to Hitchin's metric.

The regular part $\cB_\reg$ of the Hitchin base 
carries a (rigid) special \kahler structure 
in the sense of \cite{Freed:1997dp}, as follows.

The deck transformation $\sigma: \Sigma_{\phi_2} \to \Sigma_{\phi_2}$
induces an action on $H_1(\Sigma, \bbZ)$.
Let $H_1(\Sigma_{\phi_2}, \bbZ)^\pm$ denote the $\pm 1$-eigenspaces.

\begin{defn}[Charge lattice]
The \ti{charge lattice} is
\begin{equation}
    \Gamma_{\phi_2} = H_1(\Sigma_{\phi_2}, \bbZ)^- \, .
\end{equation}
\end{defn}

Let $\IP{,}$ denote the intersection pairing on $\Gamma$
and $\DIP{,}$ its inverse on $\Gamma^*_\bbR$.
These lattices make up a local system $\Gamma$ over $\cB_\reg$.
We write local formulas using a local
trivialization of $\Gamma_\bbR$ by ``$A$ and $B$ cycles''
obeying
\begin{equation}
  \IP{A^I, A^J}=0, \quad \IP{B_I, B_J}=0, \quad \IP{A^I,B_J} = \delta^I_J.
\end{equation}

\begin{defn}[Period map]
Let $\lambda$ denote the tautological (Liouville) holomorphic $1$-form
on $\Tot[K_C]$.
The \ti{period map} is the map
\begin{equation}
   Z: \Gamma_{\phi_2} \to \bbC, \qquad Z_\gamma = \oint_\gamma \lambda
\end{equation}
which we could also view as an element $Z \in \Gamma^*_\bbC$.
\end{defn}
Then we have $\de Z \in \Omega^1(\cM) \otimes \Gamma_\bbC^*$,
and we can define a $2$-form $\DIP{\de Z, \de \overline{Z}}$,
given concretely by
\begin{equation}
  \DIP{\de Z, \de \overline{Z}} = \sum_{I = 1}^{r} \de Z_{A^I} \wedge \de \overline{Z}_{B_I} - \de Z_{B_I} \wedge \de \overline{Z}_{A^I} \, .
\end{equation}

% From the periods we can assemble a $2$-form
% $\DIP{\de Z,  \de Z} \in \Omega^{2,0}(\cB_\reg)$, given concretely by
% $\DIP{\de Z,  \de Z} = \sum_{I=1}^n \de Z_{A^I} \wedge \de Z_{B_I}$.
% \begin{prop}[Lagrangian property] We have
% \begin{equation}  \label{eq:lagrangian-property}
% \DIP{\de Z, \de Z} = 0.
% \end{equation}
% \end{prop}

% Note that \eqref{eq:lagrangian-property} is automatic in case
% $\dim_\bbC \cB = 1$, but otherwise it is a nontrivial
% constraint on $Z$. The idea of the proof
% of \eqref{eq:lagrangian-property} is to consider two
% tangent vectors to $\cB$ i.e. infinitesimal variations 
% of $\Sigma$, and study the corresponding variations 
% $\delta_1, \delta_2$ of 
% the cohomology class $[\lambda] \in H^1(\Sigma,\bbC)$ 
% by integrating along arbitrary
% $1$-cycles on $\Sigma$. $\delta_1$, $\delta_2$ turn out to be
% of type $(1,0)$
% (you can get them by pairing the normal variation of $\Sigma$
% with the holomorphic symplectic form on $T^* C$)
% and thus
% $\int_\Sigma \delta_1 \wedge \delta_2 = 0$.

\begin{prop}[Positivity] 
$\DIP{\de Z, \de \overline{Z}}$ is a positive $(1,1)$-form on $\cB_\reg$. 
\end{prop}
Thus it defines
a \kahler metric on $\cB_\reg$.

% From the existence of $Z$ with the above properties
% one can deduce all the structure
% of special \kahler manifold on $\cB_\reg$. In particular,
% for any choice of linearly independent ``$A$ cycles''
% $A^1, \dots, A^r \in \Gamma^\gauge$
% with $\IP{A^I, A^J} = 0$, 
% the functions
% $a^I = Z_{A^I}$ give a local coordinate
% system, so-called ``special coordinates.''
% It is conventional to also define dual coordinates
% $a_{D,J} = Z_{B_J}$.

%  The transition maps between these local coordinate
% systems are valued in $\Sp(2r,\bbZ) \ltimes \bbC^r$.


Now we consider the fiber directions.
As we have said, $\cM_{\phi_2}$ is a space of flat $U(1)$-connections
over $\Sigma_{\phi_2}$. 
In particular, for each $\gamma \in \Gamma_{\phi_2}$ 
there is a corresponding
holonomy $\theta_\gamma: \cM_{\phi_2} \to \bbR / 2 \pi \bbZ$.
Their differentials can be assembled into
  $\de \theta \in \Omega^1(\cM) \otimes \Gamma^*_\bbR$.
If we choose a basis for $\Gamma_{\phi_2}$ 
then we get $\bbR / 2 \pi \bbZ$-valued coordinates $\theta_1, \dots, \theta_{2r}$
on $\cM_{\phi_2}$.

\begin{defn}[Semiflat metric]
The \ti{semiflat metric}  $g^\sf$ on $\cM_\reg$ is the metric 
whose \kahler form in structure $I_1$ is
\begin{equation} \label{eq:omega1-sf}
  \omega^\sf_1 = 2 \DIP{\de Z, \de \overline{Z}} - \DIP{\de \theta, \de \theta}.
\end{equation}
\end{defn}                                   

Now we can formulate the weak version of our conjecture.
It says that, away from the singular fibers, Hitchin's
metric is exponentially close to the semiflat metric.

\begin{defn}[Threshold]
Fix $\phi_2 \in \cB_\reg$. Then $\abs{\phi_2}$ is a singular flat metric on $C$, with singularities at the
zeroes of $\phi_2$. The \ti{threshold} $M(\phi_2)$
is twice the length of the shortest inextendible geodesic
(saddle connection) in this metric.\footnote{We redefined it by a factor of 2 relative to \cite{dumas2019asymptotics}; this accounts for the fact
that some factors of 2 here differ from those in the references.}
\end{defn}

\begin{conj}[Weak metric conjecture] \label{conj:weak-metric}
Fix $\phi_2 \in \cB_\reg$.
As we rescale $\phi_2 \to R^2 \phi_2$ with $R > 0$,
Hitchin's metric approaches the semiflat metric
exponentially fast:
\begin{equation}
  g = g^\sf + O(\e^{-2 R M(\phi_2)}) \, .
\end{equation}
\end{conj}


% Relative to special coordinates, \eqref{eq:omega1-sf} becomes
% \begin{equation}
% 	\omega_1^\sf = -4 \I (\im \tau)_{IJ} (\de a^I \wedge \de \bar{a}^J) - 2 \de \theta_{A^I} \wedge \de \theta_{B_I},
% \end{equation}
% where $\tau_{IJ} = \frac{\partial a_{D,I}}{\partial a^J}$.
% As we will show below, $\omega_1^\sf$ is the \kahler
% form for the \hk metric on $\cM$ in the case $G = \U(1)$.


% \subsection{The Hitchin section}

% \begin{defn}[Hitchin section, for $G = \SU(2)$]
% Choose a spin structure on $C$ and thus a line bundle
% $K_C^\half$.
% Given $\phi_2 \in \cB$ we consider the Higgs bundle
% $(E,\varphi)$:
% \begin{equation}
%   E = K_C^{\half} \oplus K_C^{-\half}, \qquad \varphi = \begin{pmatrix} 0 & -\phi_2 \\ 1 & 0 \end{pmatrix}.
% \end{equation}
% Note $\phi_2$ is a section of $\Hom(K_C^{-\half}, K_C^\half) \otimes K_C = K_C^2$ as needed, and by $1$ we mean the 
% canonical section of $\Hom(K_C^\half, K_C^{-\half}) \otimes K_C = \cO$. 
% Then $(E,\varphi)$ is a stable $(C,G)$-Higgs bundle
% lying in the fiber $\cM_{\phi_2}$.
% This gives a section of the Hitchin map for this $(C,G)$.
% \end{defn}
% There is a similar construction of sections
% of the Hitchin map for other $G$; it originates in \cite{MR1174252}.

The meaning of the conjecture is that the naive cartoon drawing of the
torus fibration is exponentially close to being the correct metric picture.
It is an instance of the Strominger-Yau-Zaslow picture of Calabi-Yau
manifolds.

This conjecture is now almost proven
\cite{mochizuki2024comparison} following earlier 
work \cite{mazzeo2019asymptotic,dumas2019asymptotics,fredrickson2020exponential}: on
the section $\theta_\gamma = 0$
it is fully proven, on the full moduli space
it is proven with the exponent $RM$ instead
of the desired $2RM$.

There is a similar conjecture for higher rank
$G$. Here too it is proven that $g$ approaches
$g^\sf$ exponentially fast
\cite{fredrickson2020exponential} but the conjectured coefficient has not been verified.


\section{Our strategy}

Now let us explain our strategy for describing the \hk metric on $\cM$.

\subsection{The other complex structures of \texorpdfstring{$\cM$}{M}}

So far we have focused on just one of the complex
structures of $\cM$, which we called $I_1$,
and its holomorphic symplectic form $\Omega_1$.
We have also said that $\cM$ is \hk, which means it has more structure.
Our construction of the metric needs to use all that structure.

The definition of \hk manifold says that 
$\cM$ has complex structures $I_1, I_2, I_3$ obeying
$I_1 I_2 = -I_2 I_1 = I_3$ and cyclic permutations,
with corresponding \kahler forms $\omega_1, \omega_2, \omega_3$.
The holomorphic symplectic forms are related to the
\kahler forms by $\Omega_1 = \omega_2 + \I \omega_3$ 
and cyclic permutations.
In fact $\cM$ has a whole family of complex structures 
$I^\zeta$, $\zeta \in \bbC\bbP^1$ (where
$I^{\zeta=0} = I_1, I^{\zeta = \I} = I_2, I^{\zeta = 1} = I_3$),
and corresponding holomorphic symplectic forms $\Omega^\zeta$.
What are they?

Given a $\SL(2)$-Higgs bundle $(E,\varphi)$ and 
solution $h$ of Hitchin's equations \eqref{eq:hitchin-equation}
there is a corresponding $1$-parameter family of flat
$\SL(2,\bbC)$-connections over $C$:
\begin{equation}
  \nabla(\zeta) = \zeta^{-1} \varphi + D + \zeta \varphi^\dagger.
\end{equation}
Let $\cM^\flat$
be the moduli space of flat reductive 
$\SL(2,\bbC)$-connections over $C$. 
% with some prescribed structure at the points of $P$ 
% ---
% e.g. in the case of generic puncture data,
% the holonomy around $p \in P$ has to be conjugate to 
% the diagonal matrix $\exp(m^\zeta_p)$ with
% \begin{equation}
% m^{\zeta}_p = \zeta^{-1} m_p^\bbC + \I m_p^\R + \zeta \overline{m}_p^\bbC. 
% \end{equation}
$\cM^\flat$ carries a complex structure $I^\flat$
and a holomorphic symplectic
structure $\Omega^{ABG}$ (Atiyah-Bott-Goldman).

\begin{prop} \label{prop:generic-structures-flat-connections} For any $\zeta \in \bbC^\times$,
the map $(E,\varphi) \mapsto \nabla(\zeta)$ identifies
\begin{equation}
  (\cM, I^\zeta, \Omega^\zeta) \simarrow (\cM^\flat, I^\flat, \Omega^{ABG}) \, .
\end{equation}
\end{prop}

So the complex structures $I^\zeta$, $\zeta \in \bbC^\times$, look very different
from $I^0$ and $I^\infty$.


\subsection{Local Darboux coordinates}

\autoref{prop:generic-structures-flat-connections} 
implies that any holomorphic function $\cX$ on $\cM^\flat$,
when applied to
the flat connection $\nabla(\zeta)$, becomes a holomorphic
function on $(\cM, I^\zeta)$.
Extending this to coordinate systems, any holomorphic 
Darboux coordinate system $\{\cX_i\}$ on $(\cM^\flat, \Omega^{ABG})$
becomes a holomorphic Darboux coordinate system on
$(\cM, I^\zeta)$.

Our aim is to calculate holomorphic Darboux
coordinates $\cX_\gamma(\zeta)$ of a given fixed Higgs bundle.
Since $\nabla(\zeta)$ varies
holomorphically with $\zeta$, the coordinates
$\cX_\gamma(\zeta)$ do as well; so they can be thought of
as functions on the \ti{twistor space} of $\cM$.

\begin{itemize}
\item {\bf Q}: Which holomorphic Darboux coordinate system on
$(\cM^\flat, \Omega^{ABG})$ will we use? 
{\bf A}: We actually
will not use just one: instead, as we move around on the
Hitchin base $\cB$ and/or vary the argument of $\zeta$, 
we will choose different coordinate
systems in different regions, separated by codimension-1 ``walls.''

\insfig{higgs-metric-6}{0.8}

\item {\bf Q}: Why do you need to have such a complicated structure? {\bf A}: Because we want to study
these coordinates through their analytic properties in the
$\zeta$-plane, and only certain coordinates will be \ti{good} in the sense of having 
simple analytic behavior
as $\zeta \to 0, \infty$; moreover, which coordinates are good changes as
we move around on $\cB$ or vary the argument of $\zeta$.

% \item Q: What does ``good'' mean? A: It means that
% as $\zeta \to 0, \infty$ they have controlled asymptotics:
% we will get as $\zeta \to 0$
% \begin{equation}
%   \cX_\gamma(\zeta) \sim c_\gamma \exp(\zeta^{-1} Z_\gamma + \I \theta_\gamma).
% \end{equation}

\item {\bf Q}: How does this help you get the metric? {\bf A}: On $(\cM, I_1)$ we 
already have the holomorphic
symplectic form $\Omega_1 = \omega_2 + \I \omega_3$. All that
is missing is the third symplectic form $\omega_1$.
Once we have holomorphic Darboux coordinate functions $\cX_\gamma(\zeta)$,
we can specialize them to say $\zeta = 1$ and get a formula
for the holomorphic symplectic form $\Omega^{\zeta = 1} = \Omega_3 = \omega_1 + \I \omega_2$; then the desired $\omega_1$ is just
$\re \Omega^{\zeta = 1}$.

\item {\bf Q}: Won't the jumping of the $\cX_\gamma(\zeta)$ at the walls
cause a problem? {\bf A}: No, the jumps are always by \ti{symplectomorphisms},
so that even though $\cX_\gamma(\zeta)$ jumps, $\Omega^\zeta$ doesn't.
\end{itemize}

% \begin{remark}[The case of $G = \U(1)$]
% A toy model for what we are doing arises in the
% case $G = \U(1)$. In that case we can easily produce holomorphic Darboux coordinates: 
% just take the $\bbC^\times$-valued \ti{holonomies} of the 
% complex flat connection $\nabla(\zeta)$, which are simply\footnote{Note that in the case
% $G = \U(1)$ we have $\Sigma = C$, so $\Gamma = H_1(C,\bbZ)$.}
% \begin{equation} \label{eq:X-sf}
%   \cX_\gamma(\zeta) = \exp \left( \zeta^{-1} Z_\gamma + \I \theta_\gamma + \zeta \overline{Z}_\gamma \right).
% \end{equation}
% These functions obey the relation
% \begin{equation}
% 	\cX_{\gamma} \cX_{\gamma'} = \cX_{\gamma + \gamma'}.
% \end{equation}
% If we choose a basis $\{\gamma_1, \dots, \gamma_{2r}\}$
% for $\Gamma$, then the corresponding functions
% $\{\cX_{\gamma_1}, \dots, \cX_{\gamma_{2r}}\}$
% give coordinates on $\cM$; by abuse of
% language we refer to the whole collection 
% $\cX_\gamma$ as a coordinate system.

% Using these coordinates at $\zeta = 1$ we get
% \begin{equation}
%  \omega_1 = \re \Omega_3 = \re \DIP{\de \log \cX, \de \log \cX} = 2 \DIP{\de Z, \de \overline{Z}} - \DIP{\de \theta, \de \theta}
% \end{equation}
% which is the semiflat \kahler form \eqref{eq:omega1-sf}.
% Thus we have proven that, in case $G = \U(1)$, Hitchin's metric $g$ agrees
% with the semiflat metric $g^\sf$ on the nose!
% \end{remark}

% We could try to do similarly for $G$ nonabelian.
% Indeed, if we choose a local section $\gamma$ of the local
% system $\Gamma$, then 
% the formula \eqref{eq:X-sf} makes good local sense,
% and the functions $\cX_\gamma(\zeta)$ so defined 
% give an honest local coordinate system
% on $\cM$. The trouble is that it is not a \ti{holomorphic}
% coordinate system in complex structure $I^\zeta$, so we 
% cannot use it to compute the \hk metric.
% Instead we will construct some
% other functions $\cX_\gamma(\zeta)$ which are true 
% holomorphic Darboux coordinates. 

% Happily, it will turn out 
% that \eqref{eq:X-sf} does not have to be abandoned completely:
% it is true ``asymptotically'', in two different senses --- 
% either as we go
% to infinity in $\cM$, or as we take $\zeta \to 0$ or
% $\zeta \to \infty$.




\section{An irregular extension} \label{sec:irregular-extension}

To go further, we introduce some of the simplest model examples,
where we can actually describe the $\cX_\gamma$ concretely. These examples
however are slightly outside our original setup: we need to allow
the Higgs bundles to have singularities. To get the very simplest
setup we will actually allow \ti{irregular} singularities.


\subsection{Definitions}

We let $C = \bbC\bbP^1$, with the usual inhomogeneous coordinate $z$, 
and use objects which are \ti{meromorphic},
with poles at $z = \infty$, rather than holomorphic.

Fix a polynomial $q$ of degree $N \in 2 \bbN + 1$.
(In fact we only need to know $q$ modulo polynomials of degree
$\le \frac{N-3}{2}$.)

\begin{defn}[Irregular Higgs bundle]
An \ti{irregular $\SL(2)$-Higgs bundle} of type $q$ is
a pair $(E,\varphi)$, where:
\begin{itemize} 
  \item $E$ is a 
meromorphic vector bundle, holomorphic away from 
$z = \infty$, 
with a valuation $\nu_\infty$ on meromorphic sections,
and a nowhere-vanishing section $\eta \in \det E$ with $\nu_\infty(\eta) = 0$,
 \item $\varphi$ is a traceless meromorphic section of $\End(E) \otimes K_C$,
 holomorphic away from $z = \infty$, obeying 
\begin{equation} \label{eq:phi2-polynomial}
  \phi_2 = -\frac12 \Tr \varphi^2 = (q(z) + l(z)) \, \de z^2
\end{equation}
with $\deg l < \frac{N}{2} - 1$,
\end{itemize}
such that for any meromorphic section $s$ of $E$, 
\begin{equation}
\nu_\infty(z^{m} s) = \nu_\infty(s) + m, \qquad \nu_\infty(\varphi \cdot s) = \nu_\infty(s) + \frac{N}{2} \, .  
\end{equation}
% such that on a neighborhood of $z = \infty$,
% after pulling back to a 4-fold cover with coordinate $z^{\frac14}$,
% there is a trivialization of $E$ with
% \begin{equation}
%   \varphi = \begin{pmatrix} 1 & 0 \\ 0 & -1 \end{pmatrix} \, \sqrt{-q(z)} \, \de z + \cdots
% \end{equation}
% where the terms in $\cdots$ are holomorphic in $z^{- \frac14}$.
% and the filtration $\nu_\infty$ is the standard one.
\end{defn}

Everything we said about ordinary $\SL(2)$-Higgs bundles has an analogue for these 
irregular ones. 
In particular:

\begin{defn}[Adapted metric]
Suppose $(E,\varphi)$ is an irregular Higgs bundle. An \ti{adapted metric} in $(E,\varphi)$ is a Hermitian metric $h$ in $E$, such that
\begin{equation}
\log h(s,s) \sim 2 \nu_\infty(s) \log \abs{z} \quad \text{ as } \abs{z} \to \infty \, . 
\end{equation}
\end{defn}

\begin{thm}
If $(E,\varphi)$ is an irregular Higgs bundle, there is an adapted metric $h$ in $E$ which obeys Hitchin's equations \eqref{eq:hitchin-equation}, and $h$ is unique up to scalar multiple. \fixme{Mochizuki}
\end{thm}

\begin{thm}
There is a moduli space $\cM(q)$ parameterizing irregular Higgs bundles of type $q$ up to equivalence.
$\cM(q)$ is a smooth manifold, of dimension $N-1$.
The formula \eqref{eq:l2-metric-def} defines a complete \hk metric on $\cM(q)$.
\fixme{Biquard-Boalch plus epsilon}
\end{thm}

So this is a continuous family of \hk spaces, parameterized by the choice of $q$.
The space of polynomials $q$ has dimension $N+1$, but we should mod out by automorphisms
of $\bbC\bbP^1$ preserving $z = \infty$ and by low-degree shifts,
so the effective number of parameters in the family 
is $(N+1) - 2 - (\frac{N-3}{2} + 1) = \frac{N-1}{2}$.

The Hitchin base $\cB(q)$ in this case is just the space of polynomials 
$P_2 = q + l$ appearing above, with $q$ fixed and $l$ varying.
It has complex dimension $\frac12(N-1)$ as it should.

\begin{example}
The case $N=3$ is the first interesting one. Here we pick
\begin{equation}
  q(z) = z^3 + \Lambda z
\end{equation}
for $\Lambda \in \bbC$. Then $\cM(q)$ is a \hk space of complex dimension $2$,
depending on the choice of $\Lambda$. \fixme{show picture of $\cB(q)$}
\end{example}


% \item There is a moduli space $\cM_{2,N}$ parameterizing good filtered
% Higgs bundles of order $N$ up to equivalence.
% \end{itemize}




\section{The coordinates} \label{sec:coordinates}

Now we describe the holomorphic coordinates $\cX_\gamma(\zeta)$ on $(\cM, I^\zeta)$
which we will use.
We consider only the case of irregular Higgs bundles
on $\bbC\bbP^1$.
(A very similar construction works in the case of
Higgs bundles with regular singularities.)


\subsection{Defining the coordinates}

Fix $\phi_2 \in \cB$ and $\zeta \in \bbC^\times$.

\begin{defn}[$\zeta$-trajectories of a quadratic differential]
A \ti{$\zeta$-trajectory} of $\phi_2$ is a path on $C$
along which
$\zeta^{-1} \sqrt{-\phi_2}$ (with either choice of sign
for $\sqrt{-\phi_2}$) is a
\ti{real} and nowhere vanishing form.
\end{defn}

\begin{prop}[$\zeta$-trajectories give a foliation] The
$\zeta$-trajectories are the leaves of a singular
foliation of $C$, with singularities at the zeroes
and poles of $\phi_2$. 
At each zero of $\phi_2$, the foliation
by $\zeta$-trajectories has a three-pronged singularity,
as shown below.
\insfig{higgs-metric-8}{0.92}
\end{prop}

\begin{prop}[Ideal triangulation determined by the
$\zeta$-trajectories] \label{prop:triangulation}
Suppose $(\phi_2, \zeta)$ is generic, in the sense that
$\zeta^{-1} Z_\gamma \notin \bbR$ for all $\gamma \in \Gamma_{\phi_2}$. Then there are
$N+2$ rays $r_i$ at infinity, such that any generic $\zeta$-trajectory is asymptotic
to one of the $r_i$. The $\zeta$-trajectories determine a 
triangulation $T(\phi_2,\zeta)$ of an $(N+2)$-gon, as indicated below.
    \insfig{a1a2tri}{0.25}
\end{prop}
% \insfig{higgs-metric-9}{0.95}
% The proof of \autoref{prop:triangulation} 
% is given in \cite{Gaiotto:2009hg},
% leaning heavily on the analysis of trajctories
% given by Strebel \cite{MR743423}.

\begin{defn}[Fock-Goncharov coordinate attached to an edge] \label{def:fg-coordinate}
Fix an interior edge $E \in T(\phi_2,\zeta)$.
$E$ determines a class $\gamma \in \Gamma_{\phi_2}$, shown
below:\footnote{More precisely, the picture shows
only the projection of $\gamma$ to $C$, and does not show
the orientation. The ambiguity can be fixed as follows:
the intersection $\IP{\gamma,\hat E}$ should be positive,
where $\hat E$ denotes one of the lifts of $E$ to $\Sigma$,
oriented so that $\lambda$ is a positive $1$-form
along $\hat E$.}
\insfig{higgs-metric-10}{1.3}
To define $\cX_\gamma(\zeta)$, we consider the connection
$\nabla(\zeta)$ restricted to the quadrilateral shown.
Its space of flat sections is a $2$-dimensional vector space $V$,
equipped with $4$ distinguished lines $\ell_i \subset V$:
$\ell_i$ consists of the flat sections
which have exponentially \ti{decaying} norm 
as we go to infinity along a leaf of $T(\phi_2,\zeta)$ in the $i$-th direction.
Said otherwise, the $\ell_i$ give $4$ points of $\bbC\bbP^1$.
We define $\cX_\gamma(\zeta)$ to be the
$\SL(2,\bbC)$-invariant cross-ratio of these $4$ points:
\begin{equation}
	\cX_\gamma(\zeta) = - \frac{(\ell_1 \wedge \ell_2)(\ell_3 \wedge \ell_4)}{(\ell_2 \wedge \ell_3) (\ell_4 \wedge \ell_1)} .
\end{equation}
This definition appears in
Fock-Goncharov \cite{MR2233852}; it is a complexification of the notion of shear coordinate.
\end{defn}

Applying \autoref{def:fg-coordinate} for all edges $E$
of $T(\phi_2,\vartheta)$ gives functions
$\cX_\gamma(\zeta)$ with $\gamma$ running over a basis for $\Gamma$. 
We extend to arbitrary $\gamma$ by requiring
$\cX_\gamma \cX_\mu = \cX_{\gamma+\mu}$.
These are local
Darboux coordinates:
\begin{equation}
  \Omega^\zeta = \DIP{\de \log \cX(\zeta), \de \log \cX(\zeta)}.
\end{equation}

\subsection{Asymptotic behavior of the coordinates}

The main asymptotic property of the coordinates $\cX_\gamma(\zeta)$ is:
\begin{conj}
Fix a point of $\cM$.
Then, as $\zeta \to 0$ along any ray,
\begin{equation}
  \cX_\gamma(\zeta) \sim \exp \left(\zeta^{-1} Z_\gamma + \I \theta_\gamma + c_\gamma \right)
\end{equation}
where the constants $c_\gamma = \int_\gamma \alpha$
for some $\alpha \in \Omega^{1,1}(\Sigma)$.
Moreover, if all $\theta_\gamma = 0$, then
all $c_\gamma = 0$, so in that case
\begin{equation}
  \cX_\gamma(\zeta) \sim \exp \left(\zeta^{-1} Z_\gamma\right).
\end{equation}
\end{conj}
(The idea: it would follow from the exact WKB method applied to the connections $\nabla(\zeta) = \zeta^{-1} \varphi + \cdots$.)

\subsection{Piecewise analytic behavior of the coordinates}

As we vary $(\phi_2, \zeta)$, the
function $\cX_\gamma(\zeta)$ is only piecewise 
smooth: it suffers
a jump whenever the triangulation $T(\phi_2, \zeta)$
changes. 
The simplest kind of jump is shown below:
\insfig{higgs-metric-11}{1.25}
This jump is associated with the ``saddle connection'' 
connecting two zeroes of $\phi_2$, 
appearing in the middle of the figure.
Such a saddle connection can only 
appear when $\zeta^{-1} Z_\mu \in \bbR_-$.
The coordinates on the two sides of the jump are related by:
\begin{equation}
  \cX_\gamma \to \cX_\gamma(1 + \cX_\mu)^{\IP{\mu,\gamma}}.
\end{equation}
A similar (but more intricate) 
phenomenon occurs when we cross a $(\phi_2, \zeta)$
for which an annulus of closed trajectories appears: 
then the $\cX_\gamma$ undergo a jump of the form
\begin{equation}
  \cX_\gamma \to \cX_\gamma(1 - \cX_\mu)^{-2 \IP{\mu,\gamma}}.
\end{equation}
Both of these are instances of the following general
structure:
\begin{equation} \label{eq:ks-map}
  \cX_\gamma \to \cX_\gamma(1 - \sigma(\mu) \cX_\mu)^{\DT(\mu) \IP{\mu,\gamma}}
\end{equation}
where for a saddle connection
we have $\DT(\mu) = +1$ and $\sigma(\mu) = -1$, while
for a closed loop we have $\DT(\mu) = -2$ and $\sigma(\mu) = +1$.

% \begin{remark}[Wall structure on $\cB$]
% Suppose we fix $\zeta$ and move around in $\cM$.
% We get an interesting structure on $\cB$:
% \insfig{higgs-metric-12}{0.8}
% Each simple discriminant point (where $\phi_2$ develops
% a double zero) emits two walls. 
% Each of these two walls carries a transformation
% of the coordinates $\cX_\gamma$, of the form
% \eqref{eq:ks-map}, with $\DT(\mu) = +1$.

% These two 
% walls are ``hyperplanes'' 
% in the sense of the special \kahler structure on $\cB_\reg$:
% they are of the form $\zeta^{-1} Z_{\pm \mu} \in \bbR_-$, where 
% $\mu$ is the vanishing cycle.
% When walls collide, they 
% can generate new walls. The new walls are also of the form
%  $Z_\mu / \zeta \in \bbR_-$ for some $\mu$, and carry
%  transformations of the form \eqref{eq:ks-map}.
%  The precise structure of the new walls is completely
% determined by the requirement that the $\cX_\gamma$
% are well defined; this is essentially an application 
% of the Kontsevich-Soibelman wall-crossing formula
% \cite{ks1}. 
% \end{remark}


% \begin{remark}[More interesting discriminant points]
% When $\cB_\sing \subset \cB$ meets itself, there's a more interesting structure of walls emanating. 
% Many things can happen; here are two:
% \insfig{higgs-metric-13}{1.15}
% At left is what happens when three zeroes of $\phi_2$ collide:
% this produces a discriminant point which emits $5$ walls.
% Each of these walls carries a transformation 
% of the form \eqref{eq:ks-map} with
% $\DT(\mu) = 1$.
% At right is the example of $G = \SU(2)$, $g_C = 0$, 
% $\abs{P}=4$: as we adjust all $m^\bbC$ to zero, so that 
% all zeroes of $\phi_2$ move onto punctures,
% the six discriminant points collapse into one.
% The resulting point emits
% walls with every rational slope, each one carrying 
% a product of two transformations 
% \eqref{eq:ks-map}, with $\DT(\mu) = 8, \DT(2 \mu) = -2$.
% \end{remark}

Fix $\phi_2$ and just let $\zeta$
vary. Then $\cX_\gamma(\zeta)$ depends on 
$\zeta$ in a piecewise-analytic way: the collection
$\{\cX_\gamma(\zeta)\}_{\gamma \in \Gamma}$ jumps 
at various rays $\ell$ in the $\zeta$-plane. 
\insfig{higgs-metric-14}{0.85}
At each such ray, the jump is a
product of transformations of the form \eqref{eq:ks-map},
where the $\mu$ in \eqref{eq:ks-map} can be any
$\mu \in \Gamma$ such that $Z_\mu / \zeta \in \bbR_-$
along $\ell$.

% \begin{example}[Finite chamber for $G = \SU(2)$, $g_C = 0$, $\abs{P} = 4$]
% In case $G = \SU(2)$, $g_C = 0$, 
% $\abs{P}=4$, at least for some choices of 
% $m^\bbC$, there exists 
% a domain $D \subset \cB$ such that, when $\phi_2 \in D$, 
% the function $\cX_\gamma(\zeta)$ jumps at exactly
% $24$ rays in the $\zeta$-plane, 
% corresponding to $24$ lattice vectors $\gamma_1, \dots, \gamma_{24} \in \Gamma$ for which 
% $\DT(\gamma_i) = 1$. For all other $\gamma \in \Gamma$
% we have $\DT(\gamma) = 0$.
% One concrete example of a $\phi_2 \in D$ is
% \begin{equation}
%   \phi_2 = \frac{z^4 - \half (z^4-1)}{(z^4-1)^2} \de z^2.
% \end{equation}
% \end{example}

% \begin{remark}[Even simpler cases]
% In ``wildly ramified'' examples (where we allow higher-order poles
% for the Higgs field) the analytic structure of the functions $\cX_\gamma(\zeta)$ 
% can be even simpler: in the simplest example, 
% we can arrange that in some domain $D \subset \cB$
% there are jumps along just $4$ rays,
% as shown in the picture above. We will discuss that example
% more below.
% \end{remark}

% \begin{remark}[Higher rank] \label{rem:higher-rank}
% So far we focused on $G = \SU(2)$.
% For $\SU(K)$ with $K>2$ the situation becomes more interesting:
% instead of studying ideal triangulations $T(\phi_2, \zeta)$ 
% one needs to study \ti{WKB spectral networks} $\cW(\vphi, \zeta)$, as defined in \cite{Gaiotto2012}.\footnote{Essentially the same graphs had appeared earlier as 
% \ti{Stokes graphs} associated to the WKB analysis of
% linear scalar $K$-th order ODEs; see in particular \cite{berk:988}.}
% Then again there is a recipe for invariants $\DT(\gamma)$
% and coordinates $\cX_\gamma(\zeta)$. This recipe however 
% depends on some conjectures:
% most importantly, to construct the $\cX_\gamma(\zeta)$ 
% one has to solve a certain linear algebra problem determined by the 
% network $\cW(\vphi,\zeta)$.
% Thus the construction has been carried out to the end in various special
% cases but not for arbitrary $(G,C,\vec{m})$ and $(\vphi, \zeta)$. In the special cases which have been worked out
% \cite{Gaiotto:2012db,Neitzke:2017yos} the $\cX_\gamma(\zeta)$
% turn out to be cluster coordinate systems on $\cM$,
% i.e. they belong to the distinguished atlas constructed
% by Fock-Goncharov in \cite{MR2233852}.
% \end{remark}

\subsection{The integral equation} \label{sec:integral-equations}

We are building up an elaborate structure,
but it is only going to be useful if it allows
us to say something concrete about
$\cX_\gamma(\zeta)$. Here is one approach:

\begin{conj}[Integral equation for $\theta_\gamma = 0$] \label{conj:integral-equation}
When all $\theta_\gamma = 0$,
\begin{equation} \label{eq:integral-equation}
  \cX_\gamma(\zeta) = \cX_\gamma^\sf(\zeta) \exp \left[ \frac{1}{4\pi \I} \sum_{\mu \in \Gamma} \DT(\mu) \IP{\gamma,\mu} \int_{Z_\mu \bbR_-} \frac{\de \zeta'}{\zeta'} \frac{\zeta' + \zeta}{\zeta'-\zeta} \log(1 - \sigma(\mu) \cX_\mu(\zeta')) \right]
\end{equation}
where 
\begin{equation}
  \cX_\gamma^\sf(\zeta) = \exp \left( \zeta^{-1} Z_\gamma + \zeta \overline{Z}_\gamma \right).
\end{equation}
The signs $\sigma(\mu) = \pm 1$ obey the relation
\begin{equation}
  \sigma(\mu) \sigma(\mu') = (-1)^{\IP{\mu,\mu'}} \sigma(\mu + \mu').
\end{equation}
\end{conj}

The functions $\cX_\gamma(\zeta)$ appear on both sides of
\eqref{eq:integral-equation}. Thus \eqref{eq:integral-equation} is an integral equation, which needs to be solved for 
the whole collection $\{\cX_\gamma(\zeta)\}_{\gamma \in \Gamma}$
at once, rather than an integral formula.

\begin{itemize}
  \item {\bf Q}: Where does this equation come from? {\bf A}: It would lead to $\cX_\gamma(\zeta)$ with
  the right analytic properties in the $\zeta$-plane: asymptotics as $\zeta \to 0,\infty$
  and jumps at the rays $Z_\mu / \zeta \in \bbR_-$ with $\DT(\mu) \neq 0$.
  The optimistic hope is that these analytic properties are strong enough
  to \ti{determine} $\cX_\gamma(\zeta)$.

  \item {\bf Q}: How do you actually solve it? {\bf A}: By iteration: pick
  $\cX_\gamma(\zeta) = \cX_\gamma^\sf(\zeta)$ as initial guess, 
  and then iterate.

  \item {\bf Q}: Why would you think that that iteration would converge?
  {\bf A1}: If all $\abs{Z_\gamma}$ are large enough, and $\DT(\mu)$ doesn't grow too fast
  as a function of $\mu$ (e.g. if only finitely many are nonzero)
  saddle-point estimates show 
  the iteration defines a contraction mapping, so it must
  converge to a (unique) fixed point.
  {\bf A2}: Actually, experimentally it seems that
  it \ti{always} converges! This is strange, and
  deserves an explanation.

  \item {\bf Q}: How does this lead to the weak
  conjecture, \autoref{conj:weak-metric}, from the previous lecture? {\bf A}: 
  If we substitute $\cX = \cX^\sf$, the log in the
  integrand is bounded above by $\e^{-2 \abs{Z_\mu}}$;
  thus we expect that the first step of the iteration is already
  suppressed by $\e^{-2 M}$ where $M$ is the minimum $\abs{Z_\mu}$
  for which $\DT(\mu) \neq 0$, and later steps
  should be further exponentially suppressed. That suggests
  that just truncating to the zeroth iteration (i.e. taking $g^\sf$)
  would already give a result exponentially close to the true metric,
  and the accuracy will improve with each iteration we take.
  In particular we can truncate to the \ti{first} iteration.
  Working this out leads to
  \begin{equation}
    g = g^\sf - \frac{2}{\pi} \sum_{\mu \in \Gamma} \DT(\mu) K_0\left( 2 \abs{Z_\mu} \right) \de \abs{Z_\mu}^2 + \cdots
  \end{equation}
  where $K_0$ is the modified Bessel function.
  Note that $K_0(x) \sim \sqrt{\frac{\pi}{2x}} \e^{-x}$,
  so $g - g^\sf$ is exponentially suppressed as $\e^{-2M}$.
  The omitted terms $\cdots$ should be of order $\e^{-4M}$.
\end{itemize}


\section{Numerical tests}

The strongest evidence supporting \autoref{conj:integral-equation} is numerical,
given in \cite{dumas2020opers}.

\subsection{The Hitchin section}

\begin{defn}[Hitchin section]
Let $\cO(\alpha)$ denote the trivial meromorphic bundle over $\bbC \bbP^1$, 
with a valuation $\nu_\infty$ given by the usual pole order at $z = \infty$ 
shifted by $-\alpha$.
Then given $\phi_2 = P_2(z) \de z^2 \in \cB(q)$ we consider the Higgs bundle
$(E,\varphi)$:
\begin{equation}
  E = \cO\left(\frac{N}{4}\right) \oplus \cO\left(-\frac{N}{4}\right), \qquad \varphi = \begin{pmatrix} 0 & -P_2 \\ 1 & 0 \end{pmatrix}.
\end{equation}
% Then $(E,\varphi)$ is a stable $(C,G)$-Higgs bundle
% lying in the fiber $\cM_{\phi_2}$.
This gives a section of the Hitchin map for $\cM(q)$.
\end{defn}
% There is a similar construction of sections
% of the Hitchin map for other $G$; it originates in \cite{MR1174252}.

% Solving Hitchin's equation \eqref{eq:hitchin-equation} 
% for this Higgs bundle is equivalent to 
% finding a harmonic map from $\bbC$ to the hyperbolic disc whose Hopf differential is $\phi_2$. Such a map exists and 
% is unique up to $\PSL(2,\bbR)$ action on the disc; its image
% turns out to be an ideal polygon with $n+2$ vertices
% \cite{HTTW}, again determined up to $\PSL(2,\bbR)$ action.

% \insfig{higgs-metric-15}{0.8}


\subsection{The \texorpdfstring{$\cX_\gamma$}{Xgamma} in an example}

% All of the analysis we described in
% \S\ref{sec:coordinates}-\S\ref{sec:integral-equations}
% carries over to this case.
% The quantities $\cX_\gamma(\zeta)$ in this case are monomials in cross-ratios $r_{ijkl}$ of $n+2$ flat sections $s_i$
% of $\nabla(\zeta)$,
% determined by their asymptotic behavior along $n+2$ rays
% approaching $z \to \infty$.\footnote{When $\zeta = 1$ these quantities have
% a particularly simple geometric meaning: the flat sections
% $s_i$ are the asymptotic vertices of the
% polygon above, so the $r_{ijkl}$ are literally
% the cross-ratios of these asymptotic vertices.
% In particular they are \ti{real}: this is a special feature
% arising for these particular Higgs bundles when 
% $\abs{\zeta}=1$.}

For example, when $N=3$, we have $5$ asymptotic rays and $\dim \cM = 2$. The $\cX_\gamma(\zeta)$ are monomials in
$2$ out of the $5$ possible cross-ratios. 
Precisely \ti{which} cross-ratios we take depends on $P_2$ and $\zeta$, as we explained.
% it is determined by a triangulation $T(\phi_2, \zeta)$ 
% of an $(n+2)$-gon, constructed similarly to the
% ideal triangulation $T(\phi_2, \zeta)$ of a punctured surface 
% $C$ constructed in \autoref{prop:triangulation}.
We take the concrete example
\begin{equation}
  P_2(z) = R^2 (z^3 - 1), \qquad R \in \bbR_+.
\end{equation}
In this case the triangulation $T(\phi_2, \zeta = 1)$ looks
like:
\insfig{higgs-metric-16}{0.8}
From this picture we can read off that the 
relevant cross-ratios are $r_{1235}$ and $r_{1345}$.

\subsection{Numerical results}

In joint work with David Dumas we 
have computed the $X_\gamma = \cX_\gamma(\zeta = 1)$ numerically
in this example, in two different ways: 
\begin{itemize}
\item by
directly solving Hitchin's equation i.e. finding
the harmonic maps,
\item
by solving the integral equations of \S\ref{sec:integral-equations}.
\end{itemize}

The github repository {\tt neitzke/stokes-numerics} contains the code we used.
Some sample output:
\begin{verbatim}
comparisons.compareClusters("A1A2", R = 0.07, scratch = True, pde_nmesh = 511)
\end{verbatim}
\begin{verbatim}
{'xarcluster': [-0.5108779665615462, -1.0],
 'fdcluster': [-0.510880773551951, -1.0000000000000009],
 'sfcluster': [-0.7023314112631698, -1.0],
 'absdiff': [2.8069904048910743e-06, 8.881784197001252e-16],
 'logdiff': [-5.4944289580305394e-06, -8.881784197001248e-16],
 'phasediff': [0.0, 0.0],
 'reldiff': [5.494428958012677e-06, 8.881784197001248e-16],
 'frames': <framedata.framedata at 0x7f1a79c6a9e0>,
 'errest': {'absode': [2.7422339558380702e-14, 6.735318507305554e-14],
  'relode': [5.3676924495678885e-14, 6.739053759474921e-14]}}
\end{verbatim}

So e.g. this says that the quantity
$X_1 = r_{1235}$ at $R = 0.07$ is approximately
$0.51088$, and the integral equation computation agrees with the PDE computation 
to this precision.

% \ti{Preliminary} numerical results for
% the quantity $X_1 = r_{1235}$ are plotted below, for $R$ ranging
% from $R = 10^{-8}$ to $R = 1$. On the left we show 
% the values of $X_1$ computed by both methods, which we call
%  $X_1^{\mathrm i}$ and $X_1^{\mathrm f}$; on the scale of that
% plot it appears as though $X_1^{\mathrm i} = X_1^{\mathrm f}$,
% as predicted by \autoref{conj:integral-equation}.
% On the right we plot the difference
% $\log(X_1^{\mathrm i}) - \log(X_1^{\mathrm f})$, 
% which never exceeds $2 \times 10^{-7}$ over the range of $R$ shown.
% We expect that this residual can be attributed to 
% numerical error in the two computations (but this remains to be 
% understood in detail.)

% \insfig{numerics-plot}{0.60}

% % {\begin{center}
% % \tiny
% % \begin{tabular}{|c|c|c|c|c|c|c|}
% % \hline
% %      $R$        & $X_1^{\mathrm{f}}$ & $X_1^{\mathrm{i}}$ & $\mathrm{logerr}(X_1)$ & $X_2^{\mathrm{f}}$ & $X_2^{\mathrm{i}}$ & $\mathrm{logerr}(X_2)$\\ 
% % \hline
% % $ 1.1000\mathrm{e-}03 $ & $-6.1724\mathrm{e-}01 $ & $-6.1724\mathrm{e-}01 $ & $ 5.7634\mathrm{e-}10 $ & $-1.0000\mathrm{e+}00 $ & $-1.0000\mathrm{e+}00 $ & $-3.4417\mathrm{e-}15 $\\ 
% % $ 1.2000\mathrm{e-}03 $ & $-6.1715\mathrm{e-}01 $ & $-6.1715\mathrm{e-}01 $ & $ 6.3652\mathrm{e-}10 $ & $-1.0000\mathrm{e+}00 $ & $-1.0000\mathrm{e+}00 $ & $ 3.7748\mathrm{e-}15 $\\ 
% % $ 1.3000\mathrm{e-}03 $ & $-6.1706\mathrm{e-}01 $ & $-6.1706\mathrm{e-}01 $ & $ 7.0317\mathrm{e-}10 $ & $-1.0000\mathrm{e+}00 $ & $-1.0000\mathrm{e+}00 $ & $ 2.2204\mathrm{e-}15 $\\ 
% % $ 1.4000\mathrm{e-}03 $ & $-6.1697\mathrm{e-}01 $ & $-6.1697\mathrm{e-}01 $ & $ 7.7037\mathrm{e-}10 $ & $-1.0000\mathrm{e+}00 $ & $-1.0000\mathrm{e+}00 $ & $ 1.9318\mathrm{e-}14 $\\ 
% % $ 1.5000\mathrm{e-}03 $ & $-6.1688\mathrm{e-}01 $ & $-6.1688\mathrm{e-}01 $ & $ 8.3383\mathrm{e-}10 $ & $-1.0000\mathrm{e+}00 $ & $-1.0000\mathrm{e+}00 $ & $ 2.2204\mathrm{e-}16 $\\ 
% % $ 1.6000\mathrm{e-}03 $ & $-6.1678\mathrm{e-}01 $ & $-6.1678\mathrm{e-}01 $ & $ 9.0331\mathrm{e-}10 $ & $-1.0000\mathrm{e+}00 $ & $-1.0000\mathrm{e+}00 $ & $-3.6304\mathrm{e-}14 $\\ 
% % $ 1.7000\mathrm{e-}03 $ & $-6.1669\mathrm{e-}01 $ & $-6.1669\mathrm{e-}01 $ & $ 9.6743\mathrm{e-}10 $ & $-1.0000\mathrm{e+}00 $ & $-1.0000\mathrm{e+}00 $ & $ 9.5479\mathrm{e-}15 $\\ 
% % $ 1.8000\mathrm{e-}03 $ & $-6.1659\mathrm{e-}01 $ & $-6.1659\mathrm{e-}01 $ & $ 1.0382\mathrm{e-}09 $ & $-1.0000\mathrm{e+}00 $ & $-1.0000\mathrm{e+}00 $ & $-4.9738\mathrm{e-}14 $\\ 
% % $ 1.9000\mathrm{e-}03 $ & $-6.1650\mathrm{e-}01 $ & $-6.1650\mathrm{e-}01 $ & $ 1.1111\mathrm{e-}09 $ & $-1.0000\mathrm{e+}00 $ & $-1.0000\mathrm{e+}00 $ & $ 4.2188\mathrm{e-}15 $\\ 
% % $ 2.0000\mathrm{e-}03 $ & $-6.1640\mathrm{e-}01 $ & $-6.1640\mathrm{e-}01 $ & $ 1.1764\mathrm{e-}09 $ & $-1.0000\mathrm{e+}00 $ & $-1.0000\mathrm{e+}00 $ & $ 5.2847\mathrm{e-}14 $\\ 
% % $ 2.1000\mathrm{e-}03 $ & $-6.1630\mathrm{e-}01 $ & $-6.1630\mathrm{e-}01 $ & $ 1.2488\mathrm{e-}09 $ & $-1.0000\mathrm{e+}00 $ & $-1.0000\mathrm{e+}00 $ & $-3.6415\mathrm{e-}14 $\\ 
% % $ 2.2000\mathrm{e-}03 $ & $-6.1620\mathrm{e-}01 $ & $-6.1620\mathrm{e-}01 $ & $ 1.3255\mathrm{e-}09 $ & $-1.0000\mathrm{e+}00 $ & $-1.0000\mathrm{e+}00 $ & $ 6.4393\mathrm{e-}15 $\\ 
% % $ 2.3000\mathrm{e-}03 $ & $-6.1610\mathrm{e-}01 $ & $-6.1610\mathrm{e-}01 $ & $ 1.4013\mathrm{e-}09 $ & $-1.0000\mathrm{e+}00 $ & $-1.0000\mathrm{e+}00 $ & $ 5.7732\mathrm{e-}15 $\\ 
% % $ 2.4000\mathrm{e-}03 $ & $-6.1600\mathrm{e-}01 $ & $-6.1600\mathrm{e-}01 $ & $ 1.4668\mathrm{e-}09 $ & $-1.0000\mathrm{e+}00 $ & $-1.0000\mathrm{e+}00 $ & $-9.7700\mathrm{e-}15 $\\ 
% % $ 2.5000\mathrm{e-}03 $ & $-6.1590\mathrm{e-}01 $ & $-6.1590\mathrm{e-}01 $ & $ 1.5448\mathrm{e-}09 $ & $-1.0000\mathrm{e+}00 $ & $-1.0000\mathrm{e+}00 $ & $-6.6613\mathrm{e-}15 $\\ 
% % $ 2.6000\mathrm{e-}03 $ & $-6.1580\mathrm{e-}01 $ & $-6.1580\mathrm{e-}01 $ & $ 1.6222\mathrm{e-}09 $ & $-1.0000\mathrm{e+}00 $ & $-1.0000\mathrm{e+}00 $ & $-2.2427\mathrm{e-}14 $\\ 
% % $ 2.8000\mathrm{e-}03 $ & $-6.1559\mathrm{e-}01 $ & $-6.1559\mathrm{e-}01 $ & $ 1.7676\mathrm{e-}09 $ & $-1.0000\mathrm{e+}00 $ & $-1.0000\mathrm{e+}00 $ & $ 1.4433\mathrm{e-}14 $\\ 
% % $ 3.0000\mathrm{e-}03 $ & $-6.1538\mathrm{e-}01 $ & $-6.1538\mathrm{e-}01 $ & $ 1.9277\mathrm{e-}09 $ & $-1.0000\mathrm{e+}00 $ & $-1.0000\mathrm{e+}00 $ & $ 1.1102\mathrm{e-}15 $\\ 
% % $ 4.0000\mathrm{e-}03 $ & $-6.1429\mathrm{e-}01 $ & $-6.1429\mathrm{e-}01 $ & $ 2.7115\mathrm{e-}09 $ & $-1.0000\mathrm{e+}00 $ & $-1.0000\mathrm{e+}00 $ & $-8.3267\mathrm{e-}15 $\\ 
% % $ 5.0000\mathrm{e-}03 $ & $-6.1314\mathrm{e-}01 $ & $-6.1314\mathrm{e-}01 $ & $ 3.5473\mathrm{e-}09 $ & $-1.0000\mathrm{e+}00 $ & $-1.0000\mathrm{e+}00 $ & $-2.3315\mathrm{e-}14 $\\ 
% % $ 6.0000\mathrm{e-}03 $ & $-6.1195\mathrm{e-}01 $ & $-6.1195\mathrm{e-}01 $ & $ 4.4102\mathrm{e-}09 $ & $-1.0000\mathrm{e+}00 $ & $-1.0000\mathrm{e+}00 $ & $-9.2149\mathrm{e-}15 $\\ 
% % $ 7.0000\mathrm{e-}03 $ & $-6.1072\mathrm{e-}01 $ & $-6.1072\mathrm{e-}01 $ & $ 5.3334\mathrm{e-}09 $ & $-1.0000\mathrm{e+}00 $ & $-1.0000\mathrm{e+}00 $ & $ 2.2204\mathrm{e-}16 $\\ 
% % $ 8.0000\mathrm{e-}03 $ & $-6.0945\mathrm{e-}01 $ & $-6.0945\mathrm{e-}01 $ & $ 6.2366\mathrm{e-}09 $ & $-1.0000\mathrm{e+}00 $ & $-1.0000\mathrm{e+}00 $ & $ 3.6415\mathrm{e-}14 $\\ 
% % $ 9.0000\mathrm{e-}03 $ & $-6.0816\mathrm{e-}01 $ & $-6.0816\mathrm{e-}01 $ & $ 7.1963\mathrm{e-}09 $ & $-1.0000\mathrm{e+}00 $ & $-1.0000\mathrm{e+}00 $ & $-7.7716\mathrm{e-}16 $\\ 
% % $ 1.0000\mathrm{e-}02 $ & $-6.0684\mathrm{e-}01 $ & $-6.0684\mathrm{e-}01 $ & $ 8.1950\mathrm{e-}09 $ & $-1.0000\mathrm{e+}00 $ & $-1.0000\mathrm{e+}00 $ & $-2.4425\mathrm{e-}15 $\\ 
% % $ 2.0000\mathrm{e-}02 $ & $-5.9259\mathrm{e-}01 $ & $-5.9259\mathrm{e-}01 $ & $ 1.8879\mathrm{e-}08 $ & $-1.0000\mathrm{e+}00 $ & $-1.0000\mathrm{e+}00 $ & $-5.6621\mathrm{e-}15 $\\ 
% % $ 3.0000\mathrm{e-}02 $ & $-5.7712\mathrm{e-}01 $ & $-5.7712\mathrm{e-}01 $ & $ 3.0802\mathrm{e-}08 $ & $-1.0000\mathrm{e+}00 $ & $-1.0000\mathrm{e+}00 $ & $-2.4425\mathrm{e-}15 $\\ 
% % $ 4.0000\mathrm{e-}02 $ & $-5.6098\mathrm{e-}01 $ & $-5.6098\mathrm{e-}01 $ & $ 4.3867\mathrm{e-}08 $ & $-1.0000\mathrm{e+}00 $ & $-1.0000\mathrm{e+}00 $ & $ 1.9540\mathrm{e-}14 $\\ 
% % $ 5.0000\mathrm{e-}02 $ & $-5.4444\mathrm{e-}01 $ & $-5.4444\mathrm{e-}01 $ & $ 5.7434\mathrm{e-}08 $ & $-1.0000\mathrm{e+}00 $ & $-1.0000\mathrm{e+}00 $ & $ 6.6613\mathrm{e-}16 $\\ 
% % $ 6.0000\mathrm{e-}02 $ & $-5.2770\mathrm{e-}01 $ & $-5.2770\mathrm{e-}01 $ & $ 7.1756\mathrm{e-}08 $ & $-1.0000\mathrm{e+}00 $ & $-1.0000\mathrm{e+}00 $ & $ 7.9936\mathrm{e-}15 $\\ 
% % $ 7.0000\mathrm{e-}02 $ & $-5.1088\mathrm{e-}01 $ & $-5.1088\mathrm{e-}01 $ & $ 8.7096\mathrm{e-}08 $ & $-1.0000\mathrm{e+}00 $ & $-1.0000\mathrm{e+}00 $ & $-1.3767\mathrm{e-}14 $\\ 
% % $ 8.0000\mathrm{e-}02 $ & $-4.9409\mathrm{e-}01 $ & $-4.9409\mathrm{e-}01 $ & $ 1.0219\mathrm{e-}07 $ & $-1.0000\mathrm{e+}00 $ & $-1.0000\mathrm{e+}00 $ & $ 1.1990\mathrm{e-}14 $\\ 
% % $ 9.0000\mathrm{e-}02 $ & $-4.7740\mathrm{e-}01 $ & $-4.7740\mathrm{e-}01 $ & $ 1.1819\mathrm{e-}07 $ & $-1.0000\mathrm{e+}00 $ & $-1.0000\mathrm{e+}00 $ & $ 1.9984\mathrm{e-}14 $\\ 
% % $ 1.0000\mathrm{e-}01 $ & $-4.6088\mathrm{e-}01 $ & $-4.6088\mathrm{e-}01 $ & $ 1.3447\mathrm{e-}07 $ & $-1.0000\mathrm{e+}00 $ & $-1.0000\mathrm{e+}00 $ & $-8.5487\mathrm{e-}15 $\\ 
% % $ 1.5000\mathrm{e-}01 $ & $-3.8221\mathrm{e-}01 $ & $-3.8221\mathrm{e-}01 $ & $ 2.2336\mathrm{e-}07 $ & $-1.0000\mathrm{e+}00 $ & $-1.0000\mathrm{e+}00 $ & $ 2.5091\mathrm{e-}14 $\\ 
% % $ 2.0000\mathrm{e-}01 $ & $-3.1227\mathrm{e-}01 $ & $-3.1227\mathrm{e-}01 $ & $ 3.2033\mathrm{e-}07 $ & $-1.0000\mathrm{e+}00 $ & $-1.0000\mathrm{e+}00 $ & $-5.3957\mathrm{e-}14 $\\ 
% % $ 2.5000\mathrm{e-}01 $ & $-2.5217\mathrm{e-}01 $ & $-2.5217\mathrm{e-}01 $ & $ 4.2295\mathrm{e-}07 $ & $-1.0000\mathrm{e+}00 $ & $-1.0000\mathrm{e+}00 $ & $-4.6221\mathrm{e-}12 $\\ 
% % $ 3.0000\mathrm{e-}01 $ & $-2.0178\mathrm{e-}01 $ & $-2.0178\mathrm{e-}01 $ & $ 5.3092\mathrm{e-}07 $ & $-1.0000\mathrm{e+}00 $ & $-1.0000\mathrm{e+}00 $ & $-6.9056\mathrm{e-}14 $\\ 
% % $ 3.5000\mathrm{e-}01 $ & $-1.6030\mathrm{e-}01 $ & $-1.6030\mathrm{e-}01 $ & $ 6.4436\mathrm{e-}07 $ & $-1.0000\mathrm{e+}00 $ & $-1.0000\mathrm{e+}00 $ & $ 6.3238\mathrm{e-}13 $\\ 
% % $ 4.0000\mathrm{e-}01 $ & $-1.2664\mathrm{e-}01 $ & $-1.2664\mathrm{e-}01 $ & $ 7.6367\mathrm{e-}07 $ & $-1.0000\mathrm{e+}00 $ & $-1.0000\mathrm{e+}00 $ & $-5.9619\mathrm{e-}14 $\\ 
% % $ 4.5000\mathrm{e-}01 $ & $-9.9628\mathrm{e-}02 $ & $-9.9628\mathrm{e-}02 $ & $ 8.9095\mathrm{e-}07 $ & $-1.0000\mathrm{e+}00 $ & $-1.0000\mathrm{e+}00 $ & $-2.3173\mathrm{e-}11 $\\ 
% % $ 5.0000\mathrm{e-}01 $ & $-7.8122\mathrm{e-}02 $ & $-7.8122\mathrm{e-}02 $ & $ 1.0189\mathrm{e-}06 $ & $-1.0000\mathrm{e+}00 $ & $-1.0000\mathrm{e+}00 $ & $-2.4425\mathrm{e-}15 $\\ 
% % $ 5.5000\mathrm{e-}01 $ & $-6.1111\mathrm{e-}02 $ & $-6.1110\mathrm{e-}02 $ & $ 1.1577\mathrm{e-}06 $ & $-1.0000\mathrm{e+}00 $ & $-1.0000\mathrm{e+}00 $ & $ 4.3698\mathrm{e-}13 $\\ 
% % $ 6.0000\mathrm{e-}01 $ & $-4.7717\mathrm{e-}02 $ & $-4.7717\mathrm{e-}02 $ & $ 1.3048\mathrm{e-}06 $ & $-1.0000\mathrm{e+}00 $ & $-1.0000\mathrm{e+}00 $ & $-3.1464\mathrm{e-}13 $\\ 
% % $ 6.5000\mathrm{e-}01 $ & $-3.7210\mathrm{e-}02 $ & $-3.7210\mathrm{e-}02 $ & $ 1.4675\mathrm{e-}06 $ & $-1.0000\mathrm{e+}00 $ & $-1.0000\mathrm{e+}00 $ & $ 9.9565\mathrm{e-}13 $\\ 
% % $ 7.0000\mathrm{e-}01 $ & $-2.8988\mathrm{e-}02 $ & $-2.8988\mathrm{e-}02 $ & $ 1.6421\mathrm{e-}06 $ & $-1.0000\mathrm{e+}00 $ & $-1.0000\mathrm{e+}00 $ & $ 2.3648\mathrm{e-}13 $\\ 
% % $ 7.5000\mathrm{e-}01 $ & $-2.2566\mathrm{e-}02 $ & $-2.2566\mathrm{e-}02 $ & $ 1.8283\mathrm{e-}06 $ & $-1.0000\mathrm{e+}00 $ & $-1.0000\mathrm{e+}00 $ & $ 2.1805\mathrm{e-}13 $\\ 
% % $ 8.0000\mathrm{e-}01 $ & $-1.7558\mathrm{e-}02 $ & $-1.7558\mathrm{e-}02 $ & $ 2.0251\mathrm{e-}06 $ & $-1.0000\mathrm{e+}00 $ & $-1.0000\mathrm{e+}00 $ & $ 6.1950\mathrm{e-}14 $\\ 
% % $ 8.5000\mathrm{e-}01 $ & $-1.3656\mathrm{e-}02 $ & $-1.3656\mathrm{e-}02 $ & $ 2.2285\mathrm{e-}06 $ & $-1.0000\mathrm{e+}00 $ & $-1.0000\mathrm{e+}00 $ & $-6.8390\mathrm{e-}14 $\\ 
% % $ 9.0000\mathrm{e-}01 $ & $-1.0618\mathrm{e-}02 $ & $-1.0618\mathrm{e-}02 $ & $ 2.4377\mathrm{e-}06 $ & $-1.0000\mathrm{e+}00 $ & $-1.0000\mathrm{e+}00 $ & $-5.7732\mathrm{e-}14 $\\ 
% % $ 9.5000\mathrm{e-}01 $ & $-8.2542\mathrm{e-}03 $ & $-8.2542\mathrm{e-}03 $ & $ 2.5983\mathrm{e-}06 $ & $-1.0000\mathrm{e+}00 $ & $-1.0000\mathrm{e+}00 $ & $-8.4377\mathrm{e-}14 $\\ 
% % \hline
% % \end{tabular}
% % \end{center}
% % }

% % \medskip 

\printbibliography

\end{document}
