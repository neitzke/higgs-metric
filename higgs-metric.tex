\documentclass[12pt,letterpaper,reqno]{article}

% \usepackage{mathtools}
\usepackage{epsfig}
\usepackage{amsmath}
\usepackage{amssymb}
\usepackage{amsthm}
\usepackage{indentfirst}
\usepackage{xspace}
\usepackage{multirow}
\usepackage{hyperref}
\usepackage{xcolor}
\usepackage{verbatim}
\usepackage[letterpaper,margin=1in,headheight=15pt]{geometry}
\usepackage{mathpazo}
\usepackage{tikz-cd}
\usepackage{booktabs}
\usepackage{framed}
\usepackage{float}
\usepackage{thmtools}
\usepackage{dashrule}
\usepackage[missing=]{gitinfo2}
\usepackage{fancyhdr}

\definecolor{darkblue}{rgb}{0.1,0.1,0.7}
\definecolor{darkred}{rgb}{0.5,0.1,0.1}
\definecolor{darkgreen}{rgb}{0.0,0.42,0.06}
\hypersetup{colorlinks=true,urlcolor=darkred,linkcolor=darkblue,citecolor=darkred}
\definecolor{shadecolor}{rgb}{0.85,0.85,0.85}

% Bibliography formatting
\usepackage[bibstyle=authoryear-comp,labeldate=false,defernumbers=true,maxnames=20,uniquename=init,dashed=false,backend=biber,sorting=none]{biblatex}

\DeclareNameAlias{sortname}{first-last}

\DeclareFieldFormat{url}{\url{#1}}
\DeclareFieldFormat[article]{pages}{#1}
\DeclareFieldFormat[inproceedings]{pages}{\lowercase{pp.}#1}
\DeclareFieldFormat[incollection]{pages}{\lowercase{pp.}#1}
\DeclareFieldFormat[article]{volume}{\textbf{#1}}
\DeclareFieldFormat[article]{number}{(#1)}
\DeclareFieldFormat[article]{title}{\MakeCapital{#1}}
\DeclareFieldFormat[inproceedings]{title}{#1}
\DeclareFieldFormat{shorthandwidth}{#1}

% Don't use "In:" in bibliography. Omit urls from journal articles.
\DeclareBibliographyDriver{article}{%
  \usebibmacro{bibindex}%
  \usebibmacro{begentry}%
  \usebibmacro{author/editor}%
  \setunit{\labelnamepunct}\newblock
  \MakeSentenceCase{\usebibmacro{title}}%
  \newunit
  \printlist{language}%
  \newunit\newblock
  \usebibmacro{byauthor}%
  \newunit\newblock
  \usebibmacro{byeditor+others}%
  \newunit\newblock
  \printfield{version}%
  \newunit\newblock
%  \usebibmacro{in:}%
  \usebibmacro{journal+issuetitle}%
  \newunit\newblock
  \printfield{note}%
  \setunit{\bibpagespunct}%
  \printfield{pages}
  \newunit\newblock
  \usebibmacro{eprint}
  \newunit\newblock
  \printfield{addendum}%
  \newunit\newblock
  \usebibmacro{pageref}%
  \usebibmacro{finentry}}

% Remove dot between volume and number in journal articles.
\renewbibmacro*{journal+issuetitle}{%
  \usebibmacro{journal}%
  \setunit*{\addspace}%
  \iffieldundef{series}
    {}
    {\newunit
     \printfield{series}%
     \setunit{\addspace}}%
  \printfield{volume}%
%  \setunit*{\adddot}%
  \printfield{number}%
  \setunit{\addcomma\space}%
  \printfield{eid}%
  \setunit{\addspace}%
  \usebibmacro{issue+date}%
  \newunit\newblock
  \usebibmacro{issue}%
  \newunit}


% Bibliography categories
\def\makebibcategory#1#2{\DeclareBibliographyCategory{#1}\defbibheading{#1}{\section*{#2}}}
\makebibcategory{books}{Books}
\makebibcategory{papers}{Refereed research papers}
\makebibcategory{chapters}{Book chapters}
\makebibcategory{conferences}{Papers in conference proceedings}
\makebibcategory{techreports}{Unpublished working papers}
\makebibcategory{bookreviews}{Book reviews}
\makebibcategory{editorials}{Editorials}
\makebibcategory{phd}{PhD thesis}
\makebibcategory{subpapers}{Submitted papers}
\makebibcategory{curpapers}{Current projects}

\setlength{\bibitemsep}{2.65pt}
\setlength{\bibhang}{.8cm}
\renewcommand{\bibfont}{\small}

\renewcommand*{\bibitem}{\addtocounter{papers}{1}\item \mbox{}\hskip-0.85cm\hbox to 0.85cm{\hfill\arabic{papers}.~~}}
\defbibenvironment{bibliography}
{\list{}
  {\setlength{\leftmargin}{\bibhang}%
   \setlength{\itemsep}{\bibitemsep}%
   \setlength{\parsep}{\bibparsep}}}
{\endlist}
{\bibitem}

\newenvironment{publications}{\section{\LARGE Publications}\label{papersstart}\vspace*{0.2cm}\small
\titlespacing{\section}{0pt}{1.5ex}{1ex}\itemsep=0.00cm
}{\label{papersend}\addtocounter{sumpapers}{-1}\refstepcounter{sumpapers}\label{sumpapers}}

\def\printbib#1{\printbibliography[category=#1,heading=#1]\lastref{sumpapers}}

% Counters for keeping track of papers
\newcounter{papers}\setcounter{papers}{0}
\newcounter{sumpapers}\setcounter{sumpapers}{0}
\def\lastref#1{\addtocounter{#1}{\value{papers}}\setcounter{papers}{0}}

% theorem environments
\declaretheoremstyle[spaceabove=0.25cm,spacebelow=0.25cm,notefont=\normalfont\bfseries, notebraces={(}{)}]{theorem}
\declaretheoremstyle[spaceabove=0.25cm,spacebelow=0.25cm,bodyfont=\normalfont,notefont=\normalfont\bfseries, notebraces={(}{)}]{noital}
\declaretheoremstyle[spaceabove=0.25cm,spacebelow=0.25cm,bodyfont=\normalfont\color{darkgreen},notefont=\normalfont\bfseries, notebraces={(}{)}]{green}
\declaretheoremstyle[spaceabove=0.25cm,spacebelow=0.25cm,bodyfont=\normalfont,notefont=\normalfont\bfseries,qed=$\qedsymbol$,notebraces={(}{)}]{proofstyle}

\declaretheorem[name=Theorem,numberwithin=section,style=theorem]{thm}
\declaretheorem[name=Proposition,sibling=thm,style=theorem]{prop}
\declaretheorem[name=Corollary,sibling=thm,style=theorem]{cor}
\declaretheorem[name=Lemma,sibling=thm,style=theorem]{lem}
\declaretheorem[name=Remark,sibling=thm,style=theorem]{remark}
\declaretheorem[name=Definition,sibling=thm,style=noital]{defn}
\declaretheorem[name=Example,sibling=thm,style=noital]{example}
\declaretheorem[name=Exercise,numberwithin=section,style=green]{exercise}
\declaretheorem[name=Proof,style=proofstyle,numbered=no]{pf}

\numberwithin{equation}{section}


% macros for convenience
\newcommand{\tops}{\texorpdfstring}

\newcommand{\nid}{\noindent}

\newcommand{\fa}{{\mathfrak a}}
\newcommand{\fp}{{\mathfrak p}}
\newcommand{\fk}{{\mathfrak k}}
\newcommand{\fg}{{\mathfrak g}}
\newcommand{\ft}{{\mathfrak t}}
\newcommand{\fh}{{\mathfrak h}}
\newcommand{\fn}{{\mathfrak n}}
\newcommand{\fq}{{\mathfrak q}}
\newcommand{\fm}{{\mathfrak m}}
\newcommand{\fr}{{\mathfrak r}}
\newcommand{\fu}{{\mathfrak u}}
\newcommand{\fG}{{\mathfrak G}}

\newcommand{\cC}{\ensuremath{\mathcal C}}
\newcommand{\cG}{\ensuremath{\mathcal G}}
\newcommand{\cB}{\ensuremath{\mathcal B}}
\newcommand{\cL}{\ensuremath{\mathcal L}}
\newcommand{\cS}{\ensuremath{\mathcal S}}
\newcommand{\cF}{\ensuremath{\mathcal F}}
\newcommand{\cK}{\ensuremath{\mathcal K}}
\newcommand{\cZ}{\ensuremath{\mathcal Z}}
\newcommand{\cM}{\ensuremath{\mathcal M}}
\newcommand{\cN}{\ensuremath{\mathcal N}}
\newcommand{\cO}{\ensuremath{\mathcal O}}
\newcommand{\cH}{\ensuremath{\mathcal H}}
\newcommand{\cX}{\ensuremath{\mathcal X}}
\newcommand{\cY}{\ensuremath{\mathcal Y}}
\newcommand{\cA}{\ensuremath{\mathcal A}}
\newcommand{\cI}{\ensuremath{\mathcal I}}

\newcommand{\R}{\ensuremath{\mathbb R}}
\newcommand{\C}{\ensuremath{\mathbb C}}
\newcommand{\PP}{\ensuremath{\mathbb P}}
\newcommand{\Z}{\ensuremath{\mathbb Z}}
\newcommand{\Q}{\ensuremath{\mathbb Q}}
\newcommand{\A}{\ensuremath{\mathbb A}}
\newcommand{\bbH}{\ensuremath{\mathbb H}}
\newcommand{\bbP}{\ensuremath{\mathbb P}}
\newcommand{\bbI}{\ensuremath{\mathbb I}}
\newcommand{\bS}{\ensuremath{\mathbb S}}

\newcommand{\half}{\ensuremath{\frac{1}{2}}}
\newcommand{\qtr}{\ensuremath{\frac{1}{4}}}
\newcommand{\bq}{{\mathbf q}}
\newcommand{\N}{{\mathcal N}}
\newcommand{\F}{{\mathcal F}}
\newcommand{\HH}{{\mathcal H}}
\newcommand{\LL}{{\mathcal L}}
\newcommand{\RR}{{\mathcal R}}
\newcommand{\V}{{\mathcal V}}
\newcommand{\dirac}{\!\!\not\!\partial}
\newcommand{\Dirac}{\!\!\not\!\!D}
\newcommand{\cE}{{\mathcal E}}
\newcommand{\vs}{\not\!v}
\newcommand{\kahler}{K\"ahler\xspace}
\newcommand{\kq}{/\!\!/}
\newcommand{\kql}[1]{/\!\!/\!\!_#1\,}
\newcommand{\hk}{hyperk\"ahler\xspace}
\newcommand{\Hk}{Hyperk\"ahler\xspace}
\newcommand{\hkq}{/\!\!/\!\!/\!\!/}
\newcommand{\hkql}[1]{/\!\!/\!\!/\!\!/\!\!_#1\,}
\newcommand{\del}{\ensuremath{\partial}}
\newcommand{\delbar}{\ensuremath{\overline{\partial}}}
\newcommand{\I}{{\mathrm i}}
\newcommand{\J}{{\mathrm j}}
\newcommand{\K}{{\mathrm k}}
\newcommand{\e}{{\mathrm e}}
\newcommand\bid{{\mathbf 1}}
\newcommand{\de}{\mathrm{d}}
\newcommand{\ab}{\mathrm{ab}}
\newcommand{\sing}{\mathrm{sing}}
\newcommand{\reg}{\mathrm{reg}}
\newcommand{\vol}{\mathrm{vol}}
\renewcommand{\sf}{\mathrm{sf}}
\newcommand{\inst}{\mathrm{inst}}
\newcommand{\eff}{\mathrm{eff}}
\newcommand{\dR}{\mathrm{dR}}
\newcommand{\closed}{\mathrm{closed}}
\newcommand{\exact}{\mathrm{exact}}
\newcommand{\gauge}{\mathrm{gauge}}
\newcommand{\flavor}{\mathrm{flavor}}

\newcommand{\abs}[1]{\lvert#1\rvert}
\newcommand{\norm}[1]{\lVert#1\rVert}
\newcommand{\IP}[1]{\langle#1\rangle}
\newcommand{\DIP}[1]{\langle\!\langle#1\rangle\!\rangle}
\newcommand{\dwrt}[1]{\frac{\partial}{\partial#1}}
\newcommand{\eps}{\epsilon}
\newcommand{\simarrow}{\xrightarrow\sim}

\newcommand{\bSigma}{{\overline \Sigma}}
\newcommand{\vphi}{{\vec\phi}}

\newcommand{\mmaref}[1]{}

\newcommand{\ti}[1]{\textit{#1}}
\newcommand{\tb}[1]{\textbf{#1}}

\DeclareMathOperator{\ad}{ad}
\DeclareMathOperator{\im}{Im}
\DeclareMathOperator{\re}{Re}
\DeclareMathOperator{\Tr}{Tr}
\DeclareMathOperator{\End}{End}
\DeclareMathOperator{\Hom}{Hom}
\DeclareMathOperator{\Aut}{Aut}
\DeclareMathOperator{\Sym}{Sym}
\DeclareMathOperator{\Lie}{Lie}
\DeclareMathOperator{\diag}{diag}
\DeclareMathOperator{\Bun}{Bun}
\DeclareMathOperator{\Vect}{Vect}
\DeclareMathOperator{\Span}{Span}
\DeclareMathOperator{\grad}{grad}
\DeclareMathOperator{\rank}{rank}
\DeclareMathOperator{\ind}{ind}
\DeclareMathOperator{\coker}{coker}
\DeclareMathOperator{\Jac}{Jac}
\DeclareMathOperator{\Pic}{Pic}
\DeclareMathOperator{\Prym}{Prym}
\DeclareMathOperator{\Hol}{Hol}
\DeclareMathOperator{\gr}{gr}
\DeclareMathOperator{\Res}{Res}
\DeclareMathOperator{\Tot}{Tot}
\DeclareMathOperator{\pdeg}{pdeg}
\DeclareMathOperator{\Gr}{Gr}

\newcommand{\SU}{\mathrm{SU}}
\newcommand{\Sp}{\mathrm{Sp}}
\newcommand{\U}{\mathrm{U}}

\newcommand{\insfig}[2]{

\medskip
\noindent
\begin{minipage}{\linewidth}

\makebox[\linewidth]{\includegraphics[keepaspectratio=true,scale=#2]{figures/#1-crop.pdf}}

\end{minipage}
\medskip

}


% \newcommand{\insfig}[2]{\begin{figure}[htbp] \centering \includegraphics[scale=#2]{figures/#1-crop.pdf} \label{fig:#1} \end{figure}}
% syntax: \insfig{name}{0.5}{caption}

\newcommand{\fixme}[1]{{\color{orange}{[#1]}}}
\newcommand{\currentposition}{{\color{blue} \noindent\makebox[\linewidth]{\hdashrule{\paperwidth}{1pt}{3mm}}}}

% \mathtoolsset{showonlyrefs}

\bibliography{higgs-metric}

\begin{document}

\pagestyle{fancy}
\lhead{{\tiny \color{gray} \tt \gitAuthorIsoDate}}
\chead{\tiny \ti{Metric on moduli of Higgs Bundles, \tb{preliminary} and \tb{incomplete} draft}}
\rhead{{\tiny \color{gray} \tt \gitAbbrevHash}}
\renewcommand{\headrulewidth}{0.5pt}


\begin{center}
\tb{Metric on moduli of Higgs Bundles} \\
Andrew Neitzke \\
\tb{Preliminary} and \tb{incomplete} draft
\end{center}

{These are the notes for a lecture series on moduli of Higgs bundles.
The latest version can be found at
\begin{center}
\small \url{http://ma.utexas.edu/users/neitzke/expos/higgs-metric.pdf}
\end{center}
% Please send corrections/improvements to
% \begin{center}
% \small \tt\href{mailto:neitzke@math.utexas.edu}{neitzke@math.utexas.edu}
% \end{center}
% or as pull requests to the source repository hosted at
% \begin{center}
% \small \url{http://github.com/neitzke/higgs-metric}
% \end{center}
}

% \tableofcontents
% \renewcommand{\listtheoremname}{Quick reference}
% \listoftheorems[onlynamed]

% \newpage


\section{Lecture 1}


\subsection{Overview}

The aim of these lectures is to describe a \ti{conjectural} 
approach to 
``more explicitly'' understanding the \hk metric $g$ on 
the moduli space $\cM$ of Higgs bundles.

The basic structure we are aiming for is as follows.
$\cM$ is a complex integrable system: this means that
it admits a holomorphic fibration $\pi: \cM \to \cB$
with complex Lagrangian fibers.
There is a complex codimension $1$ 
``singular locus'' $\cB_\sing \subset \cB$.
Let $\cB_\reg = \cB \setminus \cB_\sing$.
Over $\cB_\reg$, the fibers of $\pi$ are compact
complex tori, roughly the Jacobians of a
family of smooth \ti{spectral curves} $\Sigma_\vphi \subset T^* C$
parameterized by $\vphi \in \cB_\reg$.

The picture is that the \hk metric $g$ on $\cM$ is constructed
from two constituents:
\begin{itemize}
  \item The special \kahler structure on $\cB_\reg$, constructed from \ti{periods} of the spectral curves $\Sigma_\vphi$ ($\vphi \in \cB_\reg$),
  \item A collection of integer ``Donaldson-Thomas invariants,''
  which (in a tropical sense) count special Lagrangian discs
  in $T^*C$ with boundary on $\Sigma_\vphi$.
\end{itemize}

If one neglects the Donaldson-Thomas invariants one obtains
a simple and explicit \hk metric $g^\sf$ on $\cB_\reg$, so called
because it is flat (and translation invariant) on the torus
fibers. $g^\sf$ however does not extend to the full $\cB$.

The effect of the Donaldson-Thomas invariants
is to add corrections which break the translation invariance.
Away from $\cB_\sing$, these corrections are of the order 
$\e^{-M}$ where $M$ is the area of the smallest 
special Lagrangian disc, i.e. we predict
\begin{equation}
  g = g^\sf + O(\e^{-M}).
\end{equation}
In particular, if we follow some path to infinity in $\cM$
along which $M \to \infty$, we expect to see $g$
converging to $g^\sf$.
On the other hand, as we approach $\cB_\sing$ the effect
of these corrections is large: it smoothes out the singularity
of $g^\sf$ (partially or completely, depending whether
$\cM$ is smooth or not.)

The strategy of the lectures will be roughly:
\begin{enumerate}
  \item The moduli space $\cM$, its fibration $\cM \to \cB$ and special \kahler structure.
  \item The Donaldson-Thomas invariants.
  \item The conjectural metric construction.
  \item The available evidence that the conjecture is correct.
\end{enumerate}

\subsection{Data}

Throughout these lectures we will fix data $(G,C,\vec m)$ as follows:
\begin{itemize}
  \item a group $G = \SU(K)$ or $\U(K)$, with diagonal subgroup $T \subset G$,
  \item a compact Riemann surface $C$, equipped with a finite subset $P \subset C$,
  \item a vector $\vec{m}_p = (m^\C_p, m^\R_p) \in \C^K \oplus (\R / 2 \pi \Z)^K$ for each $p \in P$.
\end{itemize}
We write $\Tr$ for the sum of the entries of a $K$-vector.
If $G = \SU(K)$ then we require
$\Tr m_p^\C = 0$ and $\Tr m_p^\R = 0$.
If $G = \U(K)$ we require instead 
$\sum_{p \in P} \Tr m_p^\C = 0$,
$\sum_{p \in P} \Tr m_p^\R = 0$.

We require that $2g_C + \abs{P} - 2 > 0$; equivalently, 
if $C$ has genus $0$ we require $\abs{P} \ge 3$, 
and if $C$ has genus $1$ we require $\abs{P} \ge 1$.
\insfig{higgs-metric-1}{0.8}

\begin{defn}[Generic puncture data]
We say $\vec{m}$ is \ti{generic} if all $m^\C_p$
have all entries distinct. 
The generic case is the simplest
case, and for the main purposes of these lectures,
it is fine to restrict to the generic case throughout.
\end{defn}

\begin{example}[The case of $G = \SU(2)$]
A good case to keep in mind is the case $G = \SU(2)$. In that case our data reduces to a Riemann surface $C$, a finite subset $P \subset C$,
and numbers $m^\C_p \in \C$, $m^\R_p \in \R$ for each $p \in P$. The generic case is the case when all $m^\C_p \neq 0$.
\end{example}

\begin{remark} The outputs of all our constructions will 
be invariant under the symmetric group $S_K$ acting on $\vec m_p$.
\end{remark}

% \fixme{remark on irregular singularities?}

\subsection{Moduli of Higgs bundles}

Usually one would start out with ordinary Higgs bundles,
but with an eye to what will come later, we go straight
to the parabolic case. Roughly this means that we will consider
Higgs fields which, rather than being holomorphic, are
allowed to have simple poles at the points $p \in P$.

\begin{defn}[Parabolic Higgs bundles for $G = \U(K)$]
When $G = \U(K)$, a $(G, C, \vec m)$-\ti{Higgs bundle} is a pair $(E, \varphi)$, where:
\begin{itemize}
  \item $E$ is a holomorphic vector bundle of rank $K$ over $C$,
  % (reduction
  % to the parabolic subgroup whose Levi subgroup is the 
  % normalizer of $m_p^\R$),
  % parabolic weights given by $m_p^\R$ and total parabolic degree zero,
  \item $\varphi$ is a holomorphic section of $\End E \otimes K_C(P)$,
  % compatible with the parabolic structure at each $p \in P$, 
  % with generalized eigenvalues of the residue given by $m_p^\C$.
\end{itemize}
with additional ``parabolic structure'' at the points $p \in P$
as follows. Each $E_p$ carries a decreasing filtration with weights in $[0,2\pi)$, where 
\begin{equation}
  \dim \Gr_\alpha E_p = \text{multiplicity of $\alpha$ in $m^\R$}.
\end{equation}
The residue $\Res_p \varphi$
preserves the filtration on $E_p$ and thus descends to
act on the graded pieces $\Gr_\alpha E_p$, with generalized
eigenvalues determined by $m^\C$. Altogether then,
\begin{equation}
 \Gr E_p = \bigoplus_{\vec{\lambda}} E_{p,\vec\lambda}
\end{equation}
where $E_{p,\vec\lambda}$ is in grade $\lambda^\R$, and 
$\Res_p \varphi$ acts on $E_{p,\vec\lambda}$
with generalized eigenvalue $\lambda^\C$.
\end{defn}

Relative to a local holomorphic trivialization of $E$
and local coordinate on $C$, $\varphi$ would
be written as a matrix of holomorphic $1$-forms: e.g. for $G = \SU(2)$,
\begin{equation}
  \varphi = \begin{pmatrix} f_1 & f_2 \\ f_3 & -f_1 \end{pmatrix} \de z,
\end{equation}
where all $f_i$ are holomorphic functions of $z$. 

\begin{remark}[Parabolic structure in case of generic puncture data]
The case of generic puncture data is simpler:
then $E_p = \bigoplus_{\vec\lambda} E_{p,\vec\lambda}$
with all $E_{p,\vec\lambda}$ one-dimensional, $\Res_p \varphi$ 
acting by $\lambda^\C$, and filtration weight given by $\lambda^\R$.
\end{remark}

\begin{remark}[Motivation of the filtration]
One can think of the filtration as a sort of modification
of the structure of the bundle $E$ to allow fractional
orders of vanishing: given a section
$s$ of $E$, if $s(p)$ is in the weight $\alpha$ piece of $E_p$,
then we say that $s(p)$  ``vanishes to order $\frac{\alpha}{2\pi}$ at $p$''. This explains in particular why we have the strange
restriction that the weights lie in $[0,2\pi)$: changing the order
of vanishing by an integer can be accomplished as usual by multiplying sections by $z^n$.
\end{remark}

\begin{defn}[Parabolic Higgs bundles for $G = \SU(K)$]
When $G = \SU(K)$, we impose the additional condition that
$\det E$ is trivial \fixme{in the parabolic sense...} and 
$\Tr \varphi = 0$.
\end{defn}

\begin{defn}[Parabolic degree]
Let $E$ be a $(G, C, \vec m)$-Higgs bundle,
and $E' \subset E$ any holomorphic subbundle preserved by
$\varphi$. Then $E'_p$ also get a filtration with
weights in $[0,2\pi)$, and we define
\begin{equation} \label{eq:pdeg}
  \pdeg E' = \deg E' + \frac{1}{2\pi} \sum_p \sum_{\mu \in [0,2\pi)} \mu \dim \Gr^\mu E'_p.
\end{equation}
\end{defn}

The extra term in \eqref{eq:pdeg} keeps track of how $E'$
sits relative to the filtration of $E_p$.
Thus, for example, with generic puncture data and $G = \SU(2)$,
the two weights which occur in the decomposition of 
$\Gr E_p$ are of the form
$(\alpha, 2\pi-\alpha)$, for some $\alpha \le \pi$.
Now suppose $E' \subset E$ is a line subbundle.
If $E'_p \subset E_p$ is a generic line, then the contribution
to $\pdeg E'$ from $p$ will be $\frac{\alpha}{2\pi}$; the
only exception arises if $E'_p$ is the line 
$E_{p,\vec\lambda}$ with $\lambda^\R = \alpha$, in which case 
the contribution will be $1 - \frac{\alpha}{2\pi}$. 

\begin{remark}Our conditions on $m^\R$ imply that $\pdeg E \in \Z$.
\end{remark}

\begin{defn}[Parabolic stability]
We say $E$ is \ti{stable} if for all $E' \subset E$ preserved
by $\varphi$ we have
\begin{equation}
  \frac{\pdeg E'}{\rank E'} < \frac{\pdeg E}{\rank E}.
\end{equation}
We say $E$ is \ti{polystable} if it is a direct sum of
stable Higgs bundles with the same $\frac{\pdeg}{\rank}$.
\end{defn}

\begin{prop}[Moduli space of Higgs bundles exists]
There is a moduli space $\cM = \cM(G,C,\vec{m})$ parameterizing
polystable Higgs bundles $(E,\varphi)$, with $\pdeg E = 0$, 
up to equivalence. 
$\cM$ is a complex manifold away
from the locus of strictly polystable Higgs bundles.
It carries a natural holomorphic symplectic form $\Omega_1$.
\end{prop}


The holomorphic symplectic form 
$\Omega_1$ comes from the fact that variations of the
holomorphic bundle $E$ are valued in $\Omega^{0,1}(\End E)$,
while variations of the Higgs field $\varphi$ are
valued in $\Omega^{1,0}(\End E)$,
and the two are naturally dual. \fixme{with some care
around punctures}


\begin{remark}[Dimension of $\cM$]
The complex dimension of $\cM$ is
\begin{equation} \label{eq:dim-M}
  \dim_\C \cM = \begin{cases} (2g_C-2) K^2 + 2 + \abs{P} K(K-1), & \quad G = \U(K), \\ (2g_C-2) (K^2-1) + \abs{P} K(K-1), & \quad G = \SU(K). \end{cases}
\end{equation}
From now on we write $\dim_C \cM = 2r$.
\end{remark}


\begin{example}[The abelian case]
$G = \U(1)$ is the abelian case, in which $\dim_\C \cM = 2g_C$ irrespective
of $\abs{P}$. When $P = \emptyset$ we have simply $\cM = T^* \Jac C$, the cotangent
bundle to a compact complex torus. When $P \neq \emptyset$, $\cM$ is a torsor
over $T^* \Jac C$.
This is the only example of $\cM$ which is so ``linear'' in nature:
for nonabelian $G$ the space $\cM$ will be much more interesting.
\end{example}

\begin{example}[Some low-dimensional nonabelian cases] Here are a few examples:
\begin{itemize}
\item If $G = \SU(2)$ and $C$ is a genus $0$ curve with $\abs{P} = 4$,
then $\dim_\C \cM = 2$.
\item If $G = \SU(3)$
and $C$ is a genus $0$ curve with $\abs{P}=3$,
then again $\dim_\C \cM = 2$.
\item If we want to take $P = \emptyset$,
then the simplest nonabelian case is
$G = \SU(2)$ and $C$ a genus $2$ curve, in which case
$\dim_\C \cM = 6$. 
\end{itemize}
\end{example}

\begin{remark}[Restriction of Jordan form]
When some $\vec\lambda$ occurs with multiplicity
greater than $1$ in $\vec{m}_p$, there 
is a natural way of getting a subspace of $\cM$:
we can restrict the Jordan block structure of
the unipotent endomorphism $\Res_p \varphi$
acting in each $E_{p,\vec\lambda}$ (e.g. if $\vec\lambda$ has
multiplicity $2$ we can require that $\Res_p \varphi$ acts
by the scalar $\lambda^\C$ in $E_{p,\vec\lambda}$ instead of
a nontrivial Jordan block). This subspace can be considered
as a moduli space of Higgs bundles in its own right, which
has all the structure we will discuss
in the rest of these lectures; 
in particular we can get
more $2$-dimensional examples in this way.
\end{remark}



\subsection{The Hitchin map}

Now we want to exhibit $\cM$ as a complex integrable system,
i.e. a holomorphic Lagrangian fibration.

Given a Higgs bundle $(E,\varphi) \in \cM$ we can consider
the eigenvalues of $\varphi(z)$: as $z \in C$ varies
these sweep out a curve in $\Tot[K_C(P)]$, a $K$-fold cover of $C$:
\begin{equation}
  \Sigma = \{ \lambda: \det(\lambda - \varphi) = 0\} \subset \Tot[K_C(P)].
\end{equation}
The curve $\Sigma$ is an invariant of the Higgs 
bundle $(E,\varphi)$, which we call
the \ti{spectral curve}. 
The projection to $C$ induces $\rho: \Sigma \to C$ which is a $K$-fold branched covering.

Now what are all the curves $\Sigma$
we can get in this way?
We can describe them by their coefficients, ie write
\begin{equation} \label{eq:spectral-coeffs}
  \det(\lambda - \varphi) = \lambda^K + \sum_{n=1}^K \lambda^{K-n} \phi_n = 0, \qquad \phi_n \in K_C(P)^n.
\end{equation}
Then we can write $\Sigma = \Sigma_{\vphi}$ where
$\vphi = (\phi_1, \dots, \phi_n)$.
The coefficients $\vphi$ lie in the Hitchin base:
\begin{defn}[Hitchin base]
If $G = \U(K)$,
the \ti{Hitchin base} $\cB = \cB(G,C,m^\C)$ is
the space of tuples $\vphi = (\phi_1, \phi_2, \dots, \phi_K)$
where $\phi_n$ is a holomorphic section of $K_C(P)^n$,
and $m_p^\C$ controls the residues $\phi_n(p)$ via the equation\footnote{Recall that for a section of $K_C^n$ the residue at a pole is the coefficient of $\frac{(\de z)^n}{z^n}$: this is a well
defined complex number. This generalizes the case of a meromorphic $1$-form
where the residue at a pole is the coefficient of $\frac{\de z}{z}$. Said otherwise, the fiber of
$K_C(P)$ over $p \in P$ is canonically trivial,
and likewise for $K_C(P)^n$.}
\begin{equation}
  \det (\lambda - m_p^\C) = \lambda^K + \sum_{n=1}^K \lambda^{K-n} \phi_n(p).
\end{equation}
If $G = \SU(K)$ then we make the same definition except that
$\phi_1 = 0$ everywhere.
$\cB$ is a complex affine space, a torsor for the complex vector space
of $\vphi$ vanishing at $P$.
% \begin{equation}
%   \cB_0 = \begin{cases} \bigoplus_{n=1}^K H^0\left(C, K_C^n((n-1)P)\right), & G = \U(K), \\
%   \bigoplus_{n=2}^K H^0\left(C, K_C^n((n-1)P)\right), & G = \SU(K).
%   \end{cases}
% \end{equation}

\begin{defn}[Hitchin map]
The \ti{Hitchin map} is the map $\pi: \cM \to \cB$ given by
\begin{equation}
  (E, \varphi) \mapsto (\phi_1, \dots, \phi_K)
\end{equation}
where the $\phi_n$ are defined by \eqref{eq:spectral-coeffs}.
\end{defn}

\begin{prop}[Hitchin map has Lagrangian fibers] The fibers
$\cM_\vphi = \pi^{-1}(\vphi)$ 
are compact complex Lagrangian subsets of $(\cM, \Omega_1)$. (In particular, $\dim_\C \cB = \half \dim_\C \cM$.)
\end{prop}

\fixme{say why?}

We can say more precisely what the fibers are, at least 
over most of the Hitchin base:

\begin{defn}[Singular locus and smooth locus]
The \ti{singular locus} $\cB_\sing \subset \cB$ is the set of $\vphi \in \cB$
for which $\Sigma_{\vec \phi}$ is singular. $\cB_\sing$ has complex codimension $1$ in $\cB$.
The \ti{smooth locus} is $\cB_\reg = \cB \setminus \cB_\sing$.
We also let $\cM_\reg = \pi^{-1}(\cB_\reg)$.
\end{defn}

\begin{prop}[Fibers of the Hitchin map over $\cB_\reg$, crude version]
Suppose $\vphi \in \cB_\reg$. Let $\cM_\vphi = \pi^{-1}(\vphi)$.
Then:
\begin{itemize}
  \item If $G = \U(K)$, then $\cM_\vphi$ is a torsor over
$\Jac \Sigma_\vphi$.
 \item If $G = \SU(K)$, then $\cM_\vphi$ is a torsor over
$\Prym (\rho: \Sigma_\vphi \to C)$.
\end{itemize}
\end{prop}

\begin{remark}[More precise picture of the fibers
when $G = \U(K)$]
A more precise description would say exactly \ti{which} torsor 
$\cM_\vphi$ is. For example, when $G = \U(K)$,
in the case of generic puncture data,
$\cM_\vphi$ is the moduli space of parabolic line bundles
over $\Sigma_\vphi$ with parabolic degree $-K(K-1)(g_C-1)$
and parabolic weights at the $K$ preimages of $p$ given
by the $K$ components of $m^\R_p$.
\end{remark}

When $\vphi \in \cB_\sing$, $\cM_\vphi$ is compact,
but generally singular.

\begin{remark}[Concrete description of the singular locus]
What is $\cB_\sing$ concretely?
The branch locus of the covering $\rho: \Sigma_\vphi \to C$ 
is the zero locus of the discriminant $\Delta_{\vphi}$ of the equation \eqref{eq:spectral-coeffs}. $\Delta_\vphi$ 
is a holomorphic section of $K_C(P)^{K(K-1)}$;
e.g. for $G = \SU(2)$,
$\Delta_{\vphi} = - 4 \phi_2$.
$\vphi \in \cB_\reg$ iff $\Delta_\vphi$ has only simple zeroes; in this case
it has $K(K-1)(2g_C+\abs{P}-2)$ of them, and the genus of $\Sigma_\vphi$ is
  $g_\Sigma = 1 + K^2(g_C - 1) + \half K(K-1) \abs{P}$.
\end{remark}

\begin{example}[Hitchin base and spectral curves for $G = \SU(2)$]
When $G = \SU(2)$, $\cB$ is the space of meromorphic quadratic
differentials $\phi_2$ on $C$, with a pole of order $\le 2$ at 
each $p \in P$, of residue $\Res_p \phi_2 = (m_p^\C)^2$.
% It is a torsor over the space $\cB_0$ of meromorphic quadratic differentials
% with pole order $\le 1$ at each $p \in P$. 
It has complex dimension
$\dim_\C \cB = 3 g_C - 3 + \abs{P}$. 
The spectral curve for a given $\phi_2 \in \cB$
is
\begin{equation}
  \Sigma_{\phi_2} = \{\lambda: \lambda^2 + \phi_2 = 0\} \subset \Tot[K_C(P)].
\end{equation}
$\Sigma_{\phi_2}$ is a $2$-fold branched covering of $C$, branched at the zeroes of $\phi_2$.
The regular locus $\cB_\reg \subset \cB$ consists of those $\phi_2$ which have only simple zeroes ($4 g_C + 2\abs{P} - 4$ of them).
\end{example}
\end{defn}


\begin{example}[A one-dimensional Hitchin base]
Suppose $G = \SU(2)$, $\abs{P} = 4$, $g_C = 0$.
If $m^\C = 0$, $\cB$ is a complex vector space of dimension $1$,
and $\cB_\sing \subset \cB$ is the origin.
If $m^\C \neq 0$, $\cB$ is a complex affine space of dimension $1$.
For generic $m^\C$, $\cB_\sing \subset \cB$ consists
of $6$ points. The fiber $\cM_\vphi$ over any $\vphi \in \cB_\sing$ is
a nodal torus.
\end{example}

\insfig{higgs-metric-2}{0.85}

We should emphasize that this one-dimensional example can lead to the
wrong mental picture about the generic case: generally,
when $m^\C = 0$, $\cB_\sing$ is some codimension-$1$ cone inside $\cB$, and in particular,
when $\dim \cB > 1$, $\cB_\sing$ is not compact. For $m^\C \neq 0$, $\cB_\sing \subset \cB$
is not a cone anymore, but near asymptotic infinity of $\cB$,
it still looks asymptotically like a cone.



\subsection{The \hk metric}

A key fact about $\cM$ is that it carries a canonically
defined \hk metric $g$. However, $g$ is not easily
written in closed form.

To construct $g$, one needs to consider \ti{Hitchin's equation}:
given a Higgs bundle $(E,\varphi)$ this is a PDE
for a Hermitian metric $h$ in $E$, written
\begin{equation} \label{eq:hitchin-equation}
  F_{D_h} + [\varphi, \varphi^{\dagger_h}] = 0.
\end{equation}
Here $D_h$ denotes the Chern connection in $(E,h)$,
the unique $h$-unitary connection compatible with
the holomorphic structure of $E$.
\fixme{write local coordinates?}
One considers \eqref{eq:hitchin-equation} for
metrics $h$ which are smooth on $C - P$ and
have a prescribed singular behavior near each
$p \in P$.
\begin{defn}[Adapted metrics for generic puncture data] 
In the case of generic puncture data, a Hermitian metric $h$
in $E$ is \ti{adapted} if, for a holomorphic section $s$ 
where $s(p) \in E_p$ has grade $\alpha$, we have
$h(s,s) \sim \abs{z}^{\frac{2 \alpha}{2\pi}}$ near $p$.
\end{defn}
For more general puncture data the situation is a bit 
more complicated: in addition to these polynomials
we need to allow logarithmic behavior, in a way dictated
by the Jordan block structure of $\varphi$. \fixme{cite Simpson}

A basic fact is: \fixme{refs: Simpson}
\begin{thm}[Existence of harmonic metrics] \label{thm:hitchin-existence}
The equation \eqref{eq:hitchin-equation} has an
adapted solution $h$ for each $(E,\varphi)$;
this $h$ is unique up to scalar multiple.
We call $h$ the \ti{harmonic metric}.
\end{thm}

Using \autoref{thm:hitchin-existence}
one can define Hitchin's metric on $\cM$, as follows.
Given a tangent vector $v$ to $\cM$ whose norm we wish to calculate, we represent $v$ by a family of Higgs bundles
$(E_t,\varphi_t)$, with harmonic metrics $h_t$.
Identifying the underlying Hermitian bundles with a single $(E,h)$
we have an arc of unitary connections $D_t$ and skew-Hermitian 
Higgs fields $\Phi_t = \varphi_t - \varphi_t^\dagger$ 
on $(E,h)$, determined up to gauge transformations i.e. 
automorphisms of $(E,h)$.
In particular, differentiating at $t = 0$ gives a pair
\begin{equation}
\frac{\de}{\de t}\Bigg\rvert_{t=0} (D_t, \Phi_t) = 
(\dot A, \dot \Phi) \in \Omega^1(\fu(E)) \oplus \I \Omega^{1}(\fu(E)),
\end{equation}
defined up to gauge transformations.
Then the norm of $v$ is the $L^2$ norm
\begin{equation}
  g(v,v) = \int_C \norm{\dot A}^2 + \norm{\dot \Phi}^2
\end{equation}
where for $(\dot A, \dot \Phi)$ we choose the representative
\ti{minimizing} the norm.

\begin{remark}[\Hk quotient] I have not really explained
why the metric $g$ constructed in this way turns out to 
be \hk, or even \kahler. The most conceptual explanation
of this comes by viewing the construction in terms of 
an infinite-dimensional \hk quotient \fixme{cite Hitchin}.
\end{remark}

% \subsection{Hyperkahler structure}

% Although it is not obvious from the way I have described it
% here, the metric $g$ on $\cM$ is in fact \hk.
% In particular, it carries symplectic structures 
% $\omega_1$, $\omega_2$, $\omega_3$:
% \fixme{review the basic structure}
% \fixme{check signs}
% \begin{subequations} \label{eq:symplectic-forms-higgs}
% \begin{align}
%   \omega_1((\dot A_1, \dot \Phi_1),(\dot A_2, \dot \Phi_2)) &= \int_C \Tr(- \dot A_1 \wedge \dot A_2 + \dot \Phi_1 \wedge \dot \Phi_2), \\
%   \omega_2((\dot A_1, \dot \Phi_1),(\dot A_2, \dot \Phi_2)) &= \int_C \Tr(\dot \Phi_1 \wedge \star \dot A_2 - \dot A_1 \wedge \star \dot \Phi_2), \\
%   \omega_3((\dot A_1, \dot \Phi_1),(\dot A_2, \dot \Phi_2)) &= \int_C \Tr(\dot\Phi_1 \wedge \dot A_2 + \dot A_1 \wedge \dot \Phi_2).
% \end{align}
% \end{subequations}
% and complex structures
% $I_1$, $I_2$, $I_3$:
% \begin{subequations}
% \begin{align}
%  I_1(\dot A, \dot \Phi) &= (\star \dot A, - \star \dot \Phi), \\
%  I_2(\dot A, \dot \Phi) &= (- \dot \Phi, \dot A), \\
%  I_3(\dot A, \dot \Phi) &= (- \star \dot\Phi, - \star \dot A).
% \end{align}
% \end{subequations}







% In particular, $\cM$ is noncompact.

\subsection{The special \kahler structure}

The regular part of the Hitchin base $\cB_\reg$ 
carries a (rigid) special \kahler structure, as follows.

\begin{defn}[Charge lattices for $G = \U(K)$]
Suppose $G = \U(K)$ and $\vphi \in \cB_\reg$. 
Then define:
\begin{itemize}
 \item $\Gamma^\flavor = \bigoplus_{p \in P} \Gamma^{\flavor,p}$ where $\Gamma^{\flavor,p}$
 is the weight lattice of the centralizer of $m^\C_p$,
 \item $\Gamma_\vphi^\gauge = H_1(\Sigma_\vphi, \Z)$,
 % \item $\Gamma_\vphi = H_1(\Sigma_\vphi - \rho^{-1}(P), \Z)$,
 \item $\Gamma_\vphi = \left( \Gamma^\flavor \oplus H_1(\Sigma'_\vphi,\Z) \right) / \sim$
\end{itemize}
where $\Sigma'_\vphi = \Sigma_\vphi - \pi^{-1}(P)$,
and the relation $\sim$ is as follows. A point of $\pi^{-1}(p)$ with ramification index $\nu$ corresponds to a factor $\U(\nu)$ in the centralizer of $m_p^\C$. We identify a clockwise 
loop around this point with the weight of the 
determinant representation of this $\U(\nu)$ factor.
\end{defn}

\begin{example}[Charge lattices for generic puncture data]
When the puncture data is generic, we can say all this
more simply:
\begin{itemize}
 \item $\Gamma^\flavor$ is the free $\Z$-module generated
by loops around the points of $\pi^{-1}(P)$,
 \item $\Gamma_\vphi^\gauge = H_1(\Sigma_\vphi, \Z)$,
 \item $\Gamma_\vphi = H_1(\Sigma'_\vphi, \Z)$.
\end{itemize}
\end{example}

In any case, these lattices assemble into an exact 
sequence of local systems of lattices over $\cB_\reg$,
\begin{equation} \label{eq:lattice-extension}
0 \to \Gamma^\flavor \to \Gamma \to \Gamma^\gauge \to 0.
\end{equation}
$\Gamma^\gauge$ also has a nondegenerate skew pairing,
the intersection pairing on $H_1(\Sigma_\vphi, \Z)$.
We will sometimes write local formulas using a local
trivialization of $\Gamma^\gauge$ by ``$A$ and $B$ cycles''
obeying
\begin{equation}
  \IP{A^I, A^J}=0, \quad \IP{B_I, B_J}=0, \quad \IP{A^I,B_J} = \delta^I_J.
\end{equation}


\begin{defn}[Period map]
Let $\lambda$ denote the meromorphic $1$-form on $\Tot[K_C(P)]$,
induced by the tautological (Liouville) holomorphic $1$-form
on $\Tot[K_C]$.
$\lambda$ has poles along $\rho^{-1}(P)$.
The \ti{period map} is the map
\begin{equation}
   Z_\vphi: \Gamma_\vphi \to \C, \qquad Z_\vphi(\gamma) = \oint_\gamma \lambda
\end{equation}
which we could also view as an element $Z \in \Gamma^*_\C$.
\end{defn}
The restriction of $Z_\vphi$ to the image of $\Gamma^\flavor$ is constant. 
It follows that the derivative $\de Z: \Gamma \to T^* \cB_\reg$ descends to
$\de Z: \Gamma^\gauge \to T^* \cB_\reg$,
which we can also view as
\begin{equation}
  \de Z \in T^* \cB_\reg \otimes (\Gamma_\C^\gauge)^*
\end{equation}
% Note moreover the dimensional ``coincidence'' that
% $\rank \Gamma^\gauge = 2 \dim_\C \cB_{\reg}$.
Let $\IP{,}$ denote the intersection pairing on $\Gamma^\gauge$
and $\DIP{,}$ its inverse on $(\Gamma^\gauge)^*$.
Then we can define a $2$-form
$\DIP{\de Z,  \de Z} \in \Omega^{2,0}(\cB_\reg)$.
Using a local trivialization of $\Gamma^\gauge$ we have
\begin{equation}
  \DIP{\de Z,  \de Z} = \sum_{I=1}^n \de Z(A^I) \wedge \de Z(B_I).
\end{equation}


\begin{prop}[Lagrangian property] We have
\begin{equation}  \label{eq:lagrangian-property}
\DIP{\de Z, \de Z} = 0.
\end{equation}
\end{prop}

Note that \eqref{eq:lagrangian-property} is automatic in case
$\dim_\C \cB = 1$, but otherwise it is a nontrivial
constraint on $Z$. The idea of the proof
of \eqref{eq:lagrangian-property} (in the unpunctured case)
is to consider two
tangent vectors to $\cB$ i.e. infinitesimal variations 
of $\Sigma$, and study the corresponding variations 
$\delta_1, \delta_2$ of 
the cohomology class $[\lambda] \in H^1(\Sigma,\C)$ 
by integrating along arbitrary
$1$-cycles on $\Sigma$. $\delta_1$, $\delta_2$ turn out to be
of type $(1,0)$
(you can get them by pairing the normal variation of $\Sigma$
with the holomorphic symplectic form on $T^* C$)
and thus
$\int_\Sigma \delta_1 \wedge \delta_2 = 0$.

Next we define a 2-form on $\cB$ by
\begin{equation}
  \omega = \DIP{\de Z, \de \overline{Z}}
\end{equation}
(or using a basis of $\Gamma^\gauge$ with 
$\IP{A^I, B_J} = \delta^I_J$, 
$\omega = \sum_{I = 1}^{n} \de Z(A^I) \wedge \de \overline{Z}(B_I)$.)

\begin{prop}[Positivity of $\omega$] $\omega$ is a
positive $(1,1)$-form on $\cB_\reg$ (and thus it defines
a \kahler metric on $\cB_\reg$).
\end{prop}

From the existence of $Z$ with the above properties
one can deduce all the structure
of special \kahler manifold on $\cB_\reg$. In particular,
for any choice of linearly independent 
vectors $A^1, \dots, A^r \in \Gamma^\gauge$
with $\IP{A^I, A^J} = 0$, \fixme{lift} the functions
$Z_{A^1}, \dots, Z_{A^r}$ give a local coordinate
system, so-called ``special coordinates.''


%  The transition maps between these local coordinate
% systems are valued in $\Sp(2r,\Z) \ltimes \C^r$.



\subsection{The Hitchin section}

% We are almost ready to write down the semiflat metric.
% To do so, we will need a natural decomposition of
% $T \cM$ into horizontal and vertical. To define ``horizontal''
% we need to be able to compare nearby fibers $\cM_\vphi$.
% For this we use:

\begin{defn}[Hitchin section, for $G = \SU(2)$]
Choose a spin structure on $C$ and thus a line bundle
$K_C^\half$.
Given $\phi_2 \in \cB$ we consider the Higgs bundle
$(E,\varphi)$:
\begin{equation}
  E = K_C^{\half} \oplus K_C^{-\half}(-P), \qquad \varphi = \begin{pmatrix} 0 & \phi_2 \\ 1 & 0 \end{pmatrix}.
\end{equation}
Note $\phi_2$ is a section of $\Hom(K_C^{-\half}(-P), K_C^\half) \otimes K_C(P) = K_C(P)^2$ as needed, and by $1$ we mean the 
canonical section of $\Hom(K_C^\half, K_C^{-\half}(-P)) \otimes K_C(P) = \cO$. If at each $p \in P$ we set
$m^\R = (\alpha,\pi-\alpha)$ for some $\alpha$,
and $m^\C = (\sqrt{\Res_p \phi_2}, -\sqrt{\Res_p \phi_2})$
then $(E,\varphi)$ is a stable \fixme{check} 
$(C,G,\vec{m})$-Higgs bundle
lying in the fiber over $\phi_2 \in \cB$.
This gives a section of the Hitchin map.
\end{defn}
There is a similar construction for other $G$. \fixme{give refs}

Using this section we can identify each fiber $\cM_\vphi$
with the torus $\Jac \Sigma_\vphi = \Hom(\Gamma, U(1))$.
In particular, this gives local coordinates on the torus
fibers: for each $\gamma \in \Gamma_\vphi$ there is a corresponding
coordinate function $\theta_\gamma: \cM_\vphi \to \R / 2 \pi \Z$.
Their differentials can be assembled into
\begin{equation}
  \de \theta \in \Omega^1(\cM) \otimes (\Gamma^\gauge)^*.
\end{equation}
If we choose a basis for $\Gamma_\vphi$ 
then we get $\cM_\vphi \simeq (\R / 2 \pi \Z)^{2r}$.

\subsection{The semiflat metric}

% In particular there is a canonical $2$-form on the fiber

\begin{defn}[Semiflat metric]
The \ti{semiflat metric} on $\cM_\reg$ is the metric $g^\sf$ 
on $(\cM_\reg, I_1)$ with
\kahler form \fixme{check factors}
\begin{equation}
  \omega^\sf_1 = 2 \DIP{\de Z, \de \bar{Z}} - \DIP{\de \theta, \de \theta}.
\end{equation}
\end{defn}






\section{Lecture 2}

\subsection{Donaldson-Thomas invariants}


\printbibliography

\end{document}
